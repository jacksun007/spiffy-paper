\section{Implementation\label{sec:Implementation}}

We implemented a compiler that parses Spiffy annotations. (These annotations are described in Table~\ref{tab:annotation_language} of Appendix~\ref{sec:Annotation_Language})
%
%% REMOVED for the FAST submission - Ashvin
%
%% The compiler uses Python Lex-Yacc (PLY 3.4)~\cite{pythonlexyacc} as its parser generator and lexical analyzer. The grammar, written in Yacc, is based on the ANSI C grammar. The compiler is invoked with a set of C header files containing the annotated data structures (e.g., \texttt{spiffy --name Ext4 ext4\_fs.h}). It parses the annotations and the annotated structures, and ignores the rest of the source code. We verify that the boolean and integer expressions used in annotations are syntactically correct by attempting to compile the expressions using a C++ compiler.
%
The compiler generates the file system's internal representation in a symbol table, containing the definitions of all the file system metadata, their annotations, their fields (including type and symbolic name), and each of their field's annotations. Next, it detects errors such as duplicate declarations or missing required arguments. Finally, the symbol table and compiler options are exported for use by the compiler's backend. 

Spiffy's backend generates a file-system specific metadata library using Jinja2~\cite{ronacher2011jinja2}.
%%a templating engine that is typically used for generating dynamic HTML content. The code generator works by processing template filters and tags in the source template files, and the output of the compiler is a pair of C++11 source and header files that can be used by applications. Our system can be compiled as either a user space library or a Linux kernel module. 
The library can be compiled as either a user space library or as part of a Linux kernel module. We linked our generated library into the Linux kernel by porting some C++ standard containers and integrating the GNU g++ compiler into the kernel build process with some minor changes.

Every annotated structure is wrapped in a class that implements the \texttt{Entity} interface. Figure~\ref{fig:ext3_dirent_library_wrapper} shows an example structure for the Ext4 directory entry. The \texttt{name} field is initialized with its name and type for introspection, and also a reference to the structure so that it can reference \texttt{self} during parsing. We make each of the fields publicly visible by using the cast and assignment operators in the field's template class. Application programmers can thus access these fields as if they were accessing the actual C structure.

%% The simple file-system traversal application shown in Figure~\ref{fig:fs-traversal-example} uses these entity structures.

The generated library performs various types of error-checking operations. For example, the parsing of offset fields ensures that objects do not cross container boundaries, and that all variable-sized structures fit within their containers. These checks are essential if an application aims to handle file system corruption. When parsing does fail, an error code is propagated to the caller of the \texttt{parse} or \texttt{serialize} function. Spiffy guarantees that as long as all type-safety checks pass, then parsing and serialization will not cause a crash.

%% REMOVED for the FAST submission - Ashvin

%% Our path-based name resolution mechanism resolves a name in the reverse order of the path from the super block. For example, suppose the path is $A\rightarrow B_{1}\rightarrow C\rightarrow B_{2}\rightarrow D$, where each symbol is a unique structure type, and $B_{1}$and $B_{2}$ are separate instances of the same type. Structure $C$ would resolve \emph{B} as $B_{1}$, but \emph{D} would resolve \emph{B} as $B_{2}$, and not see $B_{1}$. However, it can use $C.B$ to access $B_{1}$. This mechanism is implemented by associating a path object and a parent entity with each entity. After a pointer is used to fetch and parse an entity, its path object is created by performing a shallow copy of the parent's path object, and appending a pointer to the current entity. The shallow copy increments entity reference counts, ensuring that the names in the path can be referenced correctly. When a name is not specified for a structure (in the \texttt{FSSTRUCT} annotation), the corresponding entity is not added to the path.

Currently, the fetch function always reads data from storage because we have not implemented an entity cache. This doesn't affect a tree traversal in which each entity is read once, but if a structure can be accessed using multiple paths, then it would be read multiple times. 

\paragraph*{Address Spaces}

We require annotation developers to implement the \texttt{IO} interface shown in Table~\ref{tab:library-api}. The read method maps a pointer address in an address space to a physical location on disk, and then reads a container of a given size, specified by \texttt{addr.size}, into the buffer \texttt{buf}.  \texttt{addr.offset} is used to read a container within an extent. We require a byte address space implementation so that the super block can be fetched at a fixed byte offset on disk. The super block usually contains the block size, which enables a block address space implementation.

The Ext4 file address space implementation for the \texttt{Ext4IO} class (see Figure~\ref{fig:fs-traversal-example}) requires fetching the file contents associated with an inode number. It requires reading the corresponding inode structure, converting the size and offset arguments into a list of physical block numbers, fetching these blocks into memory, and combining the blocks together. For Btrfs, we currently support the RAID address space for a single device, which only allows metadata mirroring (RAID-1). For F2FS, we support the NID address space, which maps a NID (node id) to a node block. The implementation involves a lookup to see if a valid mapping entry is the journal. If not, the mapping is obtained from the node address table.

\paragraph*{Runtime Interpretation}

Offline Spiffy applications use variants of the file-system traversal algorithm shown in Figure~\ref{fig:fs-traversal-example}. Spiffy also supports online file-system aware storage applications.  To do so, we programmed a module to perform file system interpretation at the block layer of the Linux kernel using the generated libraries. This class of file system application is typically difficult to write and error prone, since manual parsing code is needed for each block type. However, our implementation only requires a small amount of bootstrap code to support any annotated file system. The rest of the code is file-system independent.

In offline applications, the fetch function reads data from disk and parses the structure. The type of the structure is known from the pointer that is passed to the fetch function. In contrast, for online interpretation, the file system performs the read, and the module only parse the block that is read by calling \texttt{FileSystem::parse\_by\_type()}. However, it needs to know the type of the block before parsing is possible. Our runtime interpretation depends on the fact that a pointer to a metadata block must be read before the pointed-to block is read. When a pointer is encountered during the parsing a block, the module tracks the type of the pointed-to block. As a result, when the pointed-to block is read, its type is known.

Our module exports several functions, including \texttt{interpret\_read} and \texttt{interpret\_write}, that need to be placed in the I/O path to perform runtime interpretation. The module maintain a mapping between block numbers and their types. After intercepting a completed read request, it checks whether a mapping exists in the mapping table, and if so, it is a metadata block and it gets parsed. Next, \texttt{process\_pointers} is invoked with a visitor that adds (or updates) all the pointers that are found in the block into the mapping table.

When the I/O operation is a write, the module needs to determine the type of the written block. A statically allocated block can be immediately parsed because its type will not change. For example, most metadata blocks in Ext4 are statically allocated. However, in Btrfs, the super block is the only statically allocated metadata block. For dynamically allocated blocks, the block must first be labeled as unknown and its contents cached, since its type may either be unknown or have changed. Interpretation for this block is deferred until it is referenced by a block that is subsequently accessed (either read or written), and whose type is known. At that point, the module will interpret all unknown blocks that are referenced.

Since most dynamically-typed blocks are data blocks, they should be discarded immediately to reduce memory overhead. For the Btrfs file system, this is relatively easy because metadata blocks are self-identifying.  For Ext4, these blocks need to be temporarily buffered until they can be interpreted. However, we use a heuristic for Ext4 to quickly identify dynamically-typed blocks that are definitely not metadata, to reduce the memory overhead of deferred interpretation. The block is first parsed as if it were a dynamically allocated block (e.g., a directory block or extent metadata block), and if the parsing results in an error, then the block is assumed to be data and discarded. This heuristic could be used in other file systems as well because most file systems have a small number of dynamically allocated metadata block types, or their blocks are self-identifying.

%Our runtime interpretation system is currently used to read file system blocks. Supporting applications that write blocks, such as IO shepherding~\cite{Gunawi07}, will require an IO manager that can make its own IO requests, which is outside the scope of what Spiffy provides.

The module currently relies on the file system to issue \texttt{trim} operations to detect deallocation of blocks so that stale entries can be removed from the mapping table. Since file systems do not guarantee correct implementation of \texttt{trim}, the module additionally flushes out entries for dynamically allocated blocks that have not been accessed recently. This works for a caching application, but may lead to mis-classification for other runtime applications. Accurate classification can be implemented by keeping the previous versions of blocks and comparing the versions at transaction commit time. However, it comes with a higher memory overhead~\cite{Fryer2012b}.

%% This is because of write ordering. If the pointer to a block is written before the block itself is written, the written block will be marked as "unknown" until a subsequent pointer to it is written. However, since the pointer is written first, the block can never be classified with our implementation. During checkpointing, write ordering is not guaranteed. This is one of the issues with not interpreting the journal but instead the final location of metadata.

\subsection{Limitation\label{subsec:spiffy_limitation}}

Spiffy does not explicitly support concurrency and the application must deal
with synchronization manually if it is multi-threaded. We currently do 
not support online modification of file system metadata.
There are some optimization
that Spiffy has yet to implement, such as support for readahead while iterating 
through contiguous extents.

The correctness of Spiffy applications currently depend on the correct and
complete annotations. Therefore, when file system format changes, Spiffy
applications will also need to update all of the file-system specific code
that is affected by the change in semantics, as well as updating the annotations 
to reflect its current format. While this appears to involve some work, it will be
less than updating a set of manually-written tools since robust parsing
and serialization routines are already provided by the library.

Our annotation strategy makes two assumptions about file systems. We assume that the interpretation of a piece of file system metadata does not depend on the interpretation of children reached by traversing that same metadata. We also assume that any context-sensitive interpretations can be performed by a function or expression free of side effects.


