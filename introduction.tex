\section{Introduction\label{sec:Introduction}}

There are many file-system aware storage applications that bypass the virtual file system interface and operate directly on the file system image. These applications require a detailed understanding of the format of a file system, including the ability to identify, parse and traverse file system structures. These applications can operate in an offline or online context, as shown in Table~\ref{tab:storage_app_taxonomy}. Examples of offline tools include a file system checker that traverses the file system image to check the consistency of its metadata~\cite{Ma2013}, and a data recovery tool that helps recover deleted files~\cite{buckeye2006recovering}.

\begin{table}
\centering{}{\small{}}%
\begin{tabular}{|l|c|c|}
\hline 
\multicolumn{1}{|c|}{\textbf{\small{}Storage Applications}} & \textbf{\small{}Category} & \textbf{\small{}Purpose}\tabularnewline
\hline 
{\small{}Differentiated services~\cite{Mesnier2011}} & {\small{}online} & \multirow{2}{*}{{\small{}performance}}\tabularnewline
\cline{1-2} 
{\small{}Defragmentation tool} & {\small{}either} & \tabularnewline
\hline 
{\small{}File system checker~\cite{Gunawi08b}} & {\small{}either} & \multirow{4}{*}{{\small{}reliability}}\tabularnewline
\cline{1-2} 
{\small{}Data recovery tool~\cite{buckeye2006recovering}} & {\small{}either} & \tabularnewline
\cline{1-2} 
{\small{}IO shepherding~\cite{Gunawi07}} & {\small{}online} & \tabularnewline
\cline{1-2} 
{\small{}Runtime verification~\cite{Fryer2012b}} & {\small{}online} & \tabularnewline
\hline 
{\small{}File system conversion tool} & {\small{}offline} & \multirow{2}{*}{{\small{}administrative}}\tabularnewline
\cline{1-2} 
{\small{}Partition editor~\cite{gedak2012manage}} & {\small{}offline} & \tabularnewline
\hline 
{\small{}Type-specific corruption~\cite{bairavasundaram2008analyzing}} & {\small{}either} & \multirow{2}{*}{{\small{}debugging}}\tabularnewline
\cline{1-2} 
{\small{}Metadata dump tool} & {\small{}either} & \tabularnewline
\hline 
\end{tabular}
\vspace{-7pt}
\caption{\label{tab:storage_app_taxonomy}Example file-system aware storage applications. Offline applications have exclusive access to the file system; online applications operate on an in-use file system.}
\vspace{-10pt}
\end{table}

Online storage applications need to understand the file-system semantics of blocks as they are accessed at runtime (e.g., whether the block contains data or metadata, whether it belongs to a specific type of file, etc.) These applications improve the performance or reliability of a storage system by performing file-system specific processing at the storage layer. For example, differentiated storage services~\cite{Mesnier2011} improve performance by preferentially caching blocks that contain file-system metadata or the data of small files. I/O shepherding~\cite{Gunawi07} improves reliability by using file structure information to implement checksumming and replication.  Similarly, Recon~\cite{Fryer2012b} improves reliability by verifying the consistency of file-system metadata at the storage layer.

% (daniel):
% \emph{Note: it is probably this `online' stuff that is causing people to ask about concurrent modification}

Today, developers of these storage applications perform ad-hoc processing of file system metadata because most file systems do not provide the requisite library code. Even when such library code exists, its interface may not be usable by all storage applications. For example, the \texttt{libext2fs} library only supports offline interpretation of a Linux Ext3/4 file system partition; it does not support online use. Furthermore, the libraries of different file systems, even when they exist, do not provide similar interfaces. As a result, these storage applications have to be developed from scratch, or significantly rewritten for each file system, impeding the adoption of new file systems or new file-system functionality.

To make matters worse, many file systems do not provide detailed and up-to-date documentation of their metadata format. The ad-hoc processing performed by these storage applications is thus error-prone and can lead to system instability, security vulnerability, and data corruption~\cite{bangert2014nail}.  For example, \texttt{fsck} can sometimes further corrupt a file system~\cite{yang2006using}. Some storage applications reduce the amount of file-system specific code in their implementation by modifying their target file system and operating system~\cite{Mesnier2011,Gunawi07}. This approach only works for specific file systems, and can introduce its own bugs. It also requires custom system software, which may be impractical in virtual machine and cloud environments.

Our aim is to reduce the burden of developing file-system aware storage applications. To do so, we enable file system developers to specify the format of their file system using a domain-specific language so that the file system metadata can be parsed, traversed and updated correctly. We introduce Spiffy,\footnote{\textbf{Sp}ecifying and \textbf{I}nterpreting the \textbf{F}ormat of \textbf{F}iles\textbf{y}stems} a language for annotating file system data structures defined in the C language. Spiffy allows file system developers to unambiguously specify the \emph{physical} layout of the file system. The annotations handle low level details such as the encoding of specific fields, and the pointer relationships between file system structures.
We compile the annotated sources to generate a Spiffy library that provides interfaces for type-safe parsing, traversal and update of file system metadata. The library allows a developer to write actions for different pieces of file system metadata, invoking file-system specific or generic code as needed, for their offline or online application.

The generic interfaces provided by the library simplify the development of applications that work across different file systems. Consider an application that shows file-system fragmentation by plotting a histogram of the size of free extents in the file system. This application needs to traverse the file system to find and parse structures that represent free space, and then collect the extent information. With Spiffy, the application code for finding and parsing structures is similar for different file systems. File-system specific actions are only needed for collecting the extent information from the free space structures (e.g., bitmaps for Ext4 and free space extents for Btrfs).

%This separation of the development of applications from the specification and interpretation of file system formats simplifies the process of building these applications, or adapting them for new file systems.

%Because file systems share common abstractions (e.g. files, directories, inodes), there are common elements and algorithms shared between each implementation of a file-system aware application. Additionally, different applications for the \textit{same} file system include the the same low-level details needed to parse file system data structures and traverse pointers between them. To the extent that this functionality overlaps, we believe that it can be factored out into components which parse and traverse file system metadata, and applications which operate on abstractions that reflect the commonalities between file systems.

%In this paper, we take steps towards this vision by creating an annotation language to specify a file system format as well as an API, which together make it possible to generate code that can traverse a file system's metadata. The generated code allows a developer to write actions for different pieces of file system metadata, invoking file-system specific or generic code as needed. The annotations handle low level details concerning the encoding of specific fields, pointer relationships between structures, and other details required to traverse, identify and parse the metadata structures correctly and automatically.

%Unfortunately, we cannot abstract away the fact that file systems sometimes differ fundamentally in semantics in ways which make fully-generic tools impossible; for example, NTFS and HFS+ support files with multiple streams of data, in contrast to flat files in VFS file systems. The author of a tool may have to choose between treating the separate streams as separate files, treating the streams as a homogenous file, or supporting the abstraction of multi-streamed files within the tool. We ultimately envision this gap being bridged with file-system specific policies, which allow an application to choose between ways of handling differences between file systems.

The complexity of modern file systems~\cite{Lu2013} raises several challenges for our specification-based approach. Many aspects of file system structures and their relationships are not captured by their declarations in header files. First, an on-disk pointer in a file-system structure may be implicitly specified, e.g., as an integer, as shown below. The naming convention suggests that this field is a pointer, but that fact cannot be deduced from the structure definition because it is embedded in file system code.

\begin{lstlisting}
struct foo { 
	__le32 bar_block_ptr; 
};
\end{lstlisting}

Second, the interpretation of file system structures can depend on other structures. For example, the size of an inode structure in a Linux Ext3/4 file system is stored in a field within the super block that must be accessed to correctly interpret an inode block. Similarly, many structures are variable sized, with the size information being stored in other structures. Third, the semantics of metadata fields may be context-sensitive. For example, pointers inside an inode structure can refer to either directory blocks or data blocks, depending on the type of the inode. Fourth, the placement of structures on disk may be implicit in the code that operates on them (e.g., an instance of structure B optionally follows structure A) and some structures may not be declared at all (e.g., treating a buffer as an array of integers). Finally, metadata interpretation must be performed efficiently, but it is impractical to load all file-system metadata into memory for large file systems. These challenges are not addressed by existing specification tools, as discussed in Section~\ref{sec:Related_Work}.

In Spiffy, the key to specifying the relationships between file system structures is a pointer annotation that specifies that a field holds an address to a data structure on physical storage. Pointers have an address space type that indicates how the address should be mapped to the physical location. In the \texttt{struct foo} example above, this annotation would help clarify that \texttt{bar\_block\_ptr} holds an address to a structure of type \texttt{bar}, and its address space type is a (little-endian) block pointer. We expose cross-structure dependencies by using a name resolution mechanism that allows annotations to name the necessary structures unambiguously. We handle context-sensitive fields and structures by providing support for conditional types and conditionally inherited structures. We also provide support for specifying implicit fields that are computed at runtime. Last, annotations can specify the granularity at which the structures should be accessed from storage, allowing efficient data access and reducing the memory footprint of the applications.

%The central challenge of our specification-based approach is that there are many aspects of file system structures and their relationships that are not captured by their declarations in C header files. Although the headers of a file system contain the structural definitions for various metadata types, they are incomplete descriptions of the file system format because crucial information is embedded within the file system code. 
% We address this through a combination of annotations and a file-system specific library API which are designed to overcome these limitations, providing the information that is necessary for correct interpretation and traversal of these structures.

%We identify six specific types of information missing from the C declarations of file system data structures, which are supplemented by annotations.

%\noindent\textbf{File System Pointers} have integer types instead of pointer types, and the mapping between their value and the referenced content (e.g. block address, byte address, or file offset) must be specified.

%\noindent\textbf{Cross-Structure Dependencies} affect the interpretation of a dependent metadata block. For example, the size of an inode for the Ext4 file system is stored in a field within the super block that must be accessed to correctly interpret an inode block.

%\noindent\textbf{Context-Sensitive Types} are fields or data structures that can be of more than one type. For example, pointers inside an F2FS inode structure can refer to either directory blocks or data blocks.

%\noindent\textbf{Computed Fields} represent computations performed by a file system on the physical fields of a structure. For instance, converting a size field from logarithmic to integer representation, or an implicit pointer to a following optional structure.

%\noindent\textbf{Metadata Granularity} determines how file system structures relate to underlying I/O requests.

%\noindent\textbf{Constraint Checks} define sanity checks to any structure, which are useful for detecting corruption to avoid operating on bad data.

%We expand on how Spiffy addresses each of these points in~Section~\ref{subsec:Annotation-Design}. Existing specification tools do not handle all of these concerns, as discussed in~Section~\ref{sec:Related_Work}.

Together, these Spiffy features have allowed us to properly annotate three widely deployed file systems, 1) Ext4, an update-in-place file system, 2) Btrfs, a copy-on-write file system, and 3) F2FS, a log-structured file system~\cite{lee2015f2fs}. We have implemented five applications that are designed to work across file systems, a file system dump tool, a file system corruption tool, a free space display tool, a file system converter, and a storage layer service that preferentially caches data for specific users.

\begin{table*}[th!]
\centering{}%
\begin{tabular}{|c|l|l|l|c|}
\cline{2-5} 
\multicolumn{1}{c|}{} & \multicolumn{1}{c|}{\textbf{\small{}Tool}} & \multicolumn{1}{c|}{\textbf{\small{}FS}} & \multicolumn{1}{c|}{\textbf{\small{}Bug Title}} & \textbf{\small{}Closed}\tabularnewline
\hline 
% (jsun): bug\#22266: jump instruction and boot code is corrupted with
{\small{}1} & {\small{}libparted} & {\small{}Fat32} & {\small{}\#22266: jump instruction and boot code corrupted with
random bytes after fat is resized} & {\small{}2016-05}\tabularnewline
\hline 
{\small{}2} & {\small{}ntfsprogs} & {\small{}NTFS} & {\small{}Bug 723343 - Negative Number of Free Clusters in NTFS Not
Properly Interpreted} & {\small{}2014-02}\tabularnewline
\hline 
{\small{}3} & {\small{}e2fsck} & {\small{}Ext4} & {\small{}\#781110 e2fsprogs: e2fsck does not detect corruption} & {\small{}2016-05}\tabularnewline
\hline 
4 & {\small{}e2fsck} & {\small{}Ext4} & {\small{}\#760275 e2fsprogs: e2fsck corrupts Hurd filesystems} & {\small{}2015-05}\tabularnewline
\hline 
5 & {\small{}btrfsck} & {\small{}Btrfs} & {\small{}Bug 104141 - Malformed input causing crash / floating point
exception in btrfsck} & {\small{}2015-10}\tabularnewline
\hline 
%
% (jsun): REMOVED for FAST submission
%
%6 & {\small{}btrfsck} & {\small{}Btrfs} & {\small{}Bug 59541 - Btrfsck reports free space cache errors when
%using skinny extents} & {\small{}2013-06}\tabularnewline
%\hline 
\end{tabular}
\vspace{-8pt}
\caption{\label{tab:Bug-reports}Bugs due to incorrect parsing of file system
formats.}
\vspace{-8pt}
\end{table*}

\iffalse
% Ashvin - Removed for FAST submission

The rest of the paper is organized as follows. In Section~\ref{sec:Extended-Motivation}, we motivate the need for our approach. Section~\ref{sec:Approach} presents the core concepts that led to the design of the annotation language and the library API. Section~\ref{sec:File-System-Applications} describes the applications that we have implemented using the generated library. Section~\ref{sec:Implementation} describes the implementation of our system, and Section~\ref{sec:Evaluation} assesses our approach in terms of programming effort, robustness, and performance. We present related work in Section~\ref{sec:Related_Work} and discuss our conclusions in Section~\ref{sec:Conclusion}.  
%For reference, Appendix~\ref{sec:Annotation_Language} shows our file system annotation language with examples of annotated structures for the Ext4, Btrfs and F2FS file systems. 
\fi
