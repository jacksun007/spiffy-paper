%% LyX 2.2.3 created this file.  For more info, see http://www.lyx.org/.
%% Do not edit unless you really know what you are doing.
\documentclass[english]{article}
\usepackage[T1]{fontenc}
\usepackage[latin9]{inputenc}
\usepackage{babel}
\usepackage{listings}
\renewcommand{\lstlistingname}{Listing}

\begin{document}

\subsection{Example: F2FS}

F2FS is a Linux log-structured file system optimized for NAND flash
storage devices operating above a generic block interface. The on-disk
layout is divided into fixed-sized segments which represent the basic
unit of management in F2FS \cite{lee2015f2fs}. This coarse granularity
of management requires a corresponding abstraction to allow our annotations
to be sufficiently expressive. 

Managing the granularity of the annotated metadata structure is succinctly
captured in the unit parameter for the \texttt{FSSTRUCT} declaration.
As illustrated in Figure \ref{BROKEN: Ref: fig:F2FS-checkpoin=002026},
the packs within the checkpoint region contain heterogeneous typed
metadata blocks that are not necessarily in-place. Figure \ref{fig:Annotated-F2FS-Checkpoint}
shows the corresponding annotation for the checkpoint region. We define
macros to resolve external references for the checkpoint start address
to the corresponding checkpoint pack during parsing. By annotating
with the \texttt{extent} argument, we ensure that disk accesses for
a checkpoint pack do not fetch the entire segment.

\begin{figure*}
\noindent\begin{minipage}[t]{1\textwidth}%
\begin{lstlisting}[basicstyle={\footnotesize\ttfamily},tabsize=2]
#define CP_PACK1 (sb.cp_blkaddr) 
#define CP_PACK2 (sb.cp_blkaddr + (1 << sb.log_blocks_per_seg)) 
#define CP_START_ADDR ( (self.checkpoint_ver & 1) ? CP_PACK1 : CP_PACK2 ) 

VECTOR(name=f2fs_checkpoint_extent, type=f2fs_checkpoint_segment,         
		size=SEGMENT_SIZE*sb.segment_count_ckpt, count=2); 

FSSTRUCT(unit=extent, size=SEGMENT_SIZE, name=cp, base=struct f2fs_checkpoint) {
    /* Orphan Inode Blocks */         
	POINTER(name=orphan, repr=block, type=f2fs_orphan_blocks, expr=CP_START_ADDR + 1, when=HAS_ORPHANS);
    
	...

	/* Data Summary Blocks */         
	POINTER(name=normal_data_summaries, repr=block, type=normal_data_summary_blocks,                  
			expr=CP_START_ADDR + self.cp_pack_start_sum);
 
	/* Node Summary Blocks */         
	POINTER(name=node_summaries, repr=block, type=node_summary_blocks,                  
			expr=NODE_SUMM_START_ADDR, when=HAS_NODE_SUMM);

    /* Checkpoint Footer */  
	POINTER(name=footer, repr=block, type=struct f2fs_checkpoint,                  
		expr=(CP_START_ADDR + self.cp_pack_total_block_count - 1)); 

}; f2fs_checkpoint_segment 
\end{lstlisting}
%
\end{minipage}

\vspace{-3ex}
\caption{\label{fig:Annotated-F2FS-Checkpoint}Annotated F2FS Checkpoint Region
showing the use of extents of heterogeneous block structures.}
\end{figure*}

\begin{figure*}
\noindent\begin{minipage}[t]{1\textwidth}%
\begin{lstlisting}[basicstyle={\footnotesize\ttfamily},tabsize=2]
FSSTRUCT(unit=container, size=BLOCK_SIZE) {     
	u8 padding[F2FS_NODE_NUM_BYTES];     
	struct node_footer footer; 
} f2fs_block;

FSSTRUCT(base=struct f2fs_block, when=self.footer.ino == self.footer.nid) {     
	POINTER(name=i, repr=offset, type=struct f2fs_inode, expr=0); 
} inode_block;

FSSTRUCT() f2fs_nat_entry { 	
	__u8 version; /* latest version of cached nat entry */ 	
	__le32 ino;		/* inode number */

    POINTER(repr=block, type=struct f2fs_block) 	
	__le32 block_addr;	/* block address */ 
} __attribute__((packed));

FSSTRUCT(rank=container, size=BLOCK_SIZE) f2fs_nat_block { 	
	struct f2fs_nat_entry entries[NAT_ENTRY_PER_BLOCK]; 
} __attribute__((packed));

VECTOR(name=f2fs_nat_extent, type=struct f2fs_nat_block, size=SEGMENT_SIZE * sb.segment_count_nat); 
\end{lstlisting}
%
\end{minipage}

\vspace{-3ex}
\caption{\label{fig:nat_structures-1}Annotated F2FS NAT data structures. The
node footer information is used to infer whether the node block is
an inode block.}
\end{figure*}

The Node Address Table (NAT) is used as an indexing structure to locate
metadata blocks in the main (sequential-write) area and is also composed
of two NAT packs. Since the first contiguous set of indices within
a NAT pack correspond to the domain of inode numbers, we constrain
our parsing of the NAT region to this set of entries so that we only
traverse \texttt{POINTER} references to inode blocks. Figure \ref{fig:nat_structures-1}
show the relevant structures for the NAT region. Indirect and double
indirect blocks are located after parsing the inode block. 

The \texttt{node} address space is introduced to resolve node references
within the inode block to locate indirect and double indirect blocks.
This address space translation module provides a convenient mechanism
to resolve multiple translations for a given node reference (one from
each NAT pack) and preserves the goal of a tree parsing order.
\end{document}
