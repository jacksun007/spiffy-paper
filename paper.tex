% This version uses the latex2e styles, not the very ancient 2.09 stuff.
\documentclass[letterpaper,twocolumn,10pt]{article}
\usepackage{usenix,epsfig,endnotes}
\usepackage[english]{babel}
\usepackage[bookmarks=true,pdfborder={0 0 0}]{hyperref}
\usepackage{url}
\usepackage[T1]{fontenc}
\usepackage[latin9]{inputenc}
\usepackage{color}
\usepackage{array}
\usepackage{textcomp}
\usepackage{multirow}
\usepackage{amsmath}
\usepackage{amsthm}
\usepackage{graphicx}
\usepackage{makecell}
\usepackage{enumitem}

%\usepackage{titlesec}

\makeatletter
\providecommand{\tabularnewline}{\\}
\date{}

\renewcommand{\dblfloatpagefraction}{0.95}
\renewcommand{\floatpagefraction}{0.95}

% (jsun): this adds the section symbol in front of section references 
%\renewcommand*{\p@section}{\S\,}
%\renewcommand*{\p@subsection}{\S\,}

\setlength{\belowcaptionskip}{-1ex}
\usepackage{listings}
\lstset{
    basicstyle=\small\ttfamily,
    tabsize=2,
    columns=fullflexible,
    keepspaces=true,
    language=c
}

\renewcommand{\paragraph}{%
  \@startsection{paragraph}{4}%
  {\z@}{1ex \@plus 1ex \@minus .2ex}{-1em}%
  {\normalfont\normalsize\bfseries}%
}

\makeatother
\usepackage{listings}
\renewcommand{\lstlistingname}{Listing}

\begin{document}
\title{\Large \bf Spiffy: Enabling File-System Aware Storage Applications}
\author{
  {\rm Kuei Sun, Daniel Fryer, Matthew Lakier, Joseph Chu, Angela Demke Brown and Ashvin Goel}\\
  University of Toronto
%  \and
%      {\rm Daniel Fryer}\\
%      University of Toronto
%  \and
%      {\rm Matthew Lakier}\\
%      University of Toronto
%  \and
%      {\rm Angela Demke Brown}\\
%      University of Toronto
%  \and
%      {\rm Ashvin Goel}\\
%      University of Toronto
} % end author

\maketitle

\begin{abstract}
Many file-system applications such as defragmentation tools, file system checkers or data recovery tools, operate at the storage layer. Today, developers of these storage applications require detailed knowledge of the file-system format, which takes a significant amount of time to learn, often by trial and error, due to insufficient documentation or specification of the format. Furthermore, these applications perform ad-hoc processing of the file-system metadata, leading to bugs and vulnerabilities. 

% All of these hurdles impede the development of innovative file-system application by reducing potential developers to those who also work on the file systems themselves.
% (jsun): don't really like the above sentence. Not sure if it emphasized the point that we could lower 
% the bar for building fs apps by making it eaiser to work with fs metadata.

We propose Spiffy, an annotation language for specifying the on-disk format of a file system. File-system developers annotate the data structures of a file system, and we use these annotations to generate a library that allows identifying, parsing and traversing file-system metadata, providing support for both offline and online storage applications. This approach simplifies the development of storage applications that work across different file systems because it reduces the amount of file-system specific code that needs to be written.

%
% (jsun): don't actually need to itemize all of the tools, just the cool ones
%
We have written annotations for the Linux Ext4, Btrfs and F2FS file systems, and developed several applications for these file systems, including a type-specific metadata corruptor, a file system converter, and an online storage layer cache that preferentially caches files for certain users. 
Our experiments show that applications that use the library to access file system metadata can achieve good performance and are robust against file system corruption errors.

\end{abstract}

\section{Introduction\label{sec:Introduction}}

There are many file-system aware storage applications that bypass the virtual file system interface and operate directly on the file system image. These applications require a detailed understanding of the format of a file system, including the ability to identify, parse and traverse file system structures. These applications can operate in an offline or online context, as shown in Table~\ref{tab:storage_app_taxonomy}. Examples of offline tools include a file system checker that traverses the file system image to check the consistency of its metadata~\cite{Ma2013}, and a data recovery tool that helps recover deleted files~\cite{buckeye2006recovering}.

\begin{table}
\centering{}{\small{}}%
\begin{tabular}{|>{\raggedright}p{0.5\columnwidth}|c|c|}
\hline 
\multicolumn{1}{|c|}{\textbf{\small{}Storage Applications}} & \textbf{\small{}Category} & \textbf{\small{}Purpose}\tabularnewline
\hline 
{\small{}Differentiated services~\cite{Mesnier2011}} & {\small{}online} & \multirow{2}{*}{{\small{}performance}}\tabularnewline
\cline{1-2} 
{\small{}Defragmentation tool} & {\small{}either} & \tabularnewline
\hline 
{\small{}File system checker~\cite{Gunawi08b}} & {\small{}either} & \multirow{4}{*}{{\small{}reliability}}\tabularnewline
\cline{1-2} 
{\small{}Data recovery tool~\cite{buckeye2006recovering}} & {\small{}either} & \tabularnewline
\cline{1-2} 
{\small{}IO shepherding~\cite{Gunawi07}} & {\small{}online} & \tabularnewline
\cline{1-2} 
{\small{}Runtime verification~\cite{Fryer2012b}} & {\small{}online} & \tabularnewline
\hline 
{\small{}File system conversion tool} & {\small{}offline} & \multirow{2}{*}{{\small{}administrative}}\tabularnewline
\cline{1-2} 
{\small{}Partition editor~\cite{gedak2012manage}} & {\small{}offline} & \tabularnewline
\hline 
{\small{}Type-specific corruption~\cite{bairavasundaram2008analyzing}} & {\small{}either} & \multirow{2}{*}{{\small{}debugging}}\tabularnewline
\cline{1-2} 
{\small{}Metadata dump tool} & {\small{}either} & \tabularnewline
\hline 
\end{tabular}
\vspace{-7pt}
\caption{\label{tab:storage_app_taxonomy}Examples of file-system aware storage applications. Offline applications have exclusive access to the file system; online applications operate while the file system is in use.}
\vspace{-10pt}
\end{table}

Online storage applications need to understand the file-system semantics of blocks as they are accessed at runtime. For example, whether the block is a data or a metadata block, whether it belongs to a specific type of file, etc. These applications help improve the performance or reliability of a storage system by performing file-system specific processing at the storage layer. For example, differentiated storage services~\cite{Mesnier2011} improve performance by preferentially caching blocks that contain file-system metadata or the data of small files. I/O shepherding~\cite{Gunawi07} improves reliability by using file structure information to implement checksumming and replication.  Similarly, the Recon system~\cite{Fryer2012b} improves reliability by verifying the correctness of file-system metadata operations at the storage layer.

% (daniel):
% \emph{Note: it is probably this `online' stuff that is causing people to ask about concurrent modification}

Today, developers of these storage applications perform ad-hoc processing of file system metadata because most file systems do not provide the requisite library code. Even when such library code exists, its interface may not be usable by all storage applications. For example, the libext2fs library only supports offline interpretation of a Linux Ext3/4 file system partition; it does not support online use. Furthermore, the libraries of different file systems, even when they exist, do not provide similar interfaces. As a result, these storage applications have to be developed from scratch, or significantly rewritten for each file system, impeding the adoption of new file systems or new file-system functionality.

To make matters worse, many file systems do not provide detailed and up-to-date documentation of their metadata format. The ad-hoc processing performed by these storage applications is thus error-prone and can lead to system instability, security vulnerability, and data corruption~\cite{bangert2014nail}.  For example, \texttt{fsck} can sometimes further corrupt a file system~\cite{yang2006using}. Some storage applications reduce the amount of file-system specific code in their implementation by modifying their target file system and operating system~\cite{Mesnier2011,Gunawi07}. However, this approach only works for specific file systems, and it can introduce its own set of bugs. It also requires using custom system software, which may be impractical in virtual machine and cloud environments.

Our aim is to reduce the burden of developing file-system aware storage applications. Because file systems share common abstractions (e.g. files, directories, inodes), there are common elements and algorithms shared between each implementation of a file-system aware application. Additionally, different applications for the \textit{same} file system include the the same low-level details needed to parse file system data structures and traverse pointers between them. To the extent that this functionality overlaps, we believe that it can be factored out into components which parse and traverse file system metadata, and applications which operate on abstractions that reflect the commonalities between file systems.

In this paper, we take steps towards this vision by creating an annotation language to specify a file system format as well as an API, which together make it possible to generate code that can traverse a file system's metadata. The generated code allows a developer to write actions for different pieces of file system metadata, invoking file-system specific or generic code as needed. The annotations handle low level details concerning the encoding of specific fields, pointer relationships between structures, and other details required to traverse, identify and parse the metadata structures correctly and automatically.

Unfortunately, we cannot abstract away the fact that file systems sometimes differ fundamentally in semantics in ways which make fully-generic tools impossible; for example, NTFS and HFS+ support files with multiple streams of data, in contrast to flat files in VFS file systems. The author of a tool may have to choose between treating the separate streams as separate files, treating the streams as a homogenous file, or supporting the abstraction of multi-streamed files within the tool. We ultimately envision this gap being bridged with file-system specific policies, which allow an application to choose between ways of handling differences between file systems.

%Our ultimate goal is to provide a way to declaratively map file system data structures onto higher level abstractions, to ease development of file systems as well as other applications that operate directly on file system metadata.

%so that the file system metadata can be identified, parsed and traversed correctly and automatically. 

%This simplifies the process of mapping file-system specific types (e.g. a struct ext4_inode, or a btrfs_inode_item) to a higher level abstraction (e.g. "an inode").

%Application developers can thus focus on the logic of their applications, and write file-system specific policies more easily.

%For example, these policies might cache or replicate specific types of metadata or file data.
%This separation of the development of applications from the specification and interpretation of file system formats simplifies the process of building these applications, or adapting them for new file systems.


We introduce Spiffy,\footnote{\textbf{Sp}ecifying and \textbf{I}nterpreting the \textbf{F}ormat of \textbf{F}iles\textbf{y}stems} a language for annotating file system data structures defined in the C language. The annotations augment existing data structure definitions, so that the relationships between file system structures can be explicitly and concisely stated. Spiffy allows file system developers to unambiguously specify the \emph{physical} layout of the file system. We compile the annotated source code to produce a library that enables type-safe parsing and traversal of file system metadata, both for offline or online applications. The generic interfaces provided by the library simplify the development of file-system aware applications, making it easier to write applications that work across different file systems.

% The complexity of modern file systems~\cite{Lu2013} raises several challenges for our specification-based approach. There are many aspects of file system structures and their relationships that are not captured by their declarations in header files. First, an on-disk pointer in a file-system structure may be implicitly specified as an integer, as shown below. The naming convention suggests that this field is a pointer, but that fact cannot be deduced from the structure definition; it requires higher-level knowledge from the application programmer.

% \begin{lstlisting}
% struct foo { 
% 	__le32 bar_block_ptr; 
% };
% \end{lstlisting}

% Second, the interpretation of file system structures can depend on other structures. For example, the size of an inode structure in a Linux Ext3/4 file system is stored in a field within the super block that must be accessed to correctly interpret an inode block. Similarly, many structures are variable sized, with the size information being stored in other structures. Third, the semantics of metadata fields may be context-sensitive. For example, pointers inside an F2FS inode structure can refer to either directory blocks or data blocks, depending on the type of the inode. Fourth, the placement of structures on disk may be implicit in the code that operates on them (e.g., an instance of structure B optionally follows structure A) and some structures may not be declared at all (e.g., treating a buffer as an array of ints). Finally, metadata interpretation must be performed efficiently, but it is impractical to load all file-system metadata into memory for large file systems. These challenges are not addressed by existing specification tools, as discussed in~Section Section~\ref{sec:Related_Work}.

% In Spiffy, the key to specifying the relationships between file system structures is a pointer annotation that specifies that a field holds an address to a data structure on physical storage. Pointers have an address space type that indicates how the address should be mapped to the physical location. In the \texttt{struct foo} example above, this annotation would help clarify that \texttt{bar\_block\_ptr} holds an address to a structure of type \texttt{bar}, and its address space type is a (little-endian) block pointer. We expose cross-structure dependencies by using a name resolution mechanism that allows annotations to name the necessary structures unambiguously, and by using a dependency tracking mechanism that ensures that the referenced structures are valid. We handle context-sensitive fields and structures by providing support for conditional types and conditionally inherited structures. We also provide support for specifying implicit fields that are computed at runtime. Last, annotations can specify the granularity at which the structures should be accessed from storage, allowing data to be accessed efficiently, and reducing the memory footprint of the applications.

The central challenge of our specification-based approach is that there are many aspects of file system structures and their relationships that are not captured by their declarations in C header files. Although the headers of a file system contain the structural definitions for various metadata types, they are incomplete descriptions of the file system format because crucial information is embedded within the file system code. 
% We address this through a combination of annotations and a file-system specific library API which are designed to overcome these limitations, providing the information that is necessary for correct interpretation and traversal of these structures.

We identify six specific types of information missing from the C declarations of file system data structures, which are supplemented by annotations.

\noindent\textbf{File System Pointers} have integer types instead of pointer types, and the mapping between their value and the referenced content (e.g. block address, byte address, or file offset) must be specified.

\noindent\textbf{Cross-Structure Dependencies} affect the interpretation of a dependent metadata block. For example, the size of an inode for the Ext4 file system is stored in a field within the super block that must be accessed to correctly interpret an inode block.

\noindent\textbf{Context-Sensitive Types} are fields or data structures that can be of more than one type. For example, pointers inside an F2FS inode structure can refer to either directory blocks or data blocks.

\noindent\textbf{Computed Fields} represent computations performed by a file system on the physical fields of a structure. For instance, converting a size field from logarithmic to integer representation, or an implicit pointer to a following optional structure.

\noindent\textbf{Metadata Granularity} determines how file system structures relate to underlying I/O requests.

\noindent\textbf{Constraint Checks} define sanity checks to any structure, which are useful for detecting corruption to avoid operating on bad data.

We expand on how Spiffy addresses each of these points in~Section~\ref{subsec:Annotation-Design}. Existing specification tools do not handle all of these concerns, as discussed in~Section~\ref{sec:Related_Work}.

Our annotation strategy makes two assumptions about file systems. We assume that the interpretation of a piece of file system metadata does not depend on the interpretation of children reached by traversing that same metadata. We also assume that any context-sensitive interpretations can be performed by a function or expression free of side effects.

We have annotated three file systems, 1) Ext4, an update-in-place file system, 2) Btrfs, a copy-on-write file system, and 3) F2FS, a log-structured file system~\cite{lee2015f2fs}. We have implemented four applications that are designed to work across file systems, a file system dump tool, a free space reporting tool, a file system converter, and a storage layer service that preferentially caches data for specific users.

\begin{table*}[th!]
\centering{}%
\begin{tabular}{|c|l|l|l|c|}
\cline{2-5} 
\multicolumn{1}{c|}{} & \multicolumn{1}{c|}{\textbf{\small{}Tool}} & \multicolumn{1}{c|}{\textbf{\small{}FS}} & \multicolumn{1}{c|}{\textbf{\small{}Bug Title}} & \textbf{\small{}Closed}\tabularnewline
\hline 
% (jsun): bug\#22266: jump instruction and boot code is corrupted with
{\small{}1} & {\small{}libparted} & {\small{}Fat32} & {\small{}\#22266: jump instruction and boot code corrupted with
random bytes after fat is resized} & {\small{}2016-05}\tabularnewline
\hline 
{\small{}2} & {\small{}ntfsprogs} & {\small{}NTFS} & {\small{}Bug 723343 - Negative Number of Free Clusters in NTFS Not
Properly Interpreted} & {\small{}2014-02}\tabularnewline
\hline 
{\small{}3} & {\small{}e2fsck} & {\small{}Ext4} & {\small{}\#781110 e2fsprogs: e2fsck does not detect corruption} & {\small{}2016-05}\tabularnewline
\hline 
4 & {\small{}e2fsck} & {\small{}Ext4} & {\small{}\#760275 e2fsprogs: e2fsck corrupts Hurd filesystems} & {\small{}2015-05}\tabularnewline
\hline 
5 & {\small{}btrfsck} & {\small{}Btrfs} & {\small{}Bug 104141 - Malformed input causing crash / floating point
exception in btrfsck} & {\small{}2015-10}\tabularnewline
\hline 
6 & {\small{}btrfsck} & {\small{}Btrfs} & {\small{}Bug 59541 - Btrfsck reports free space cache errors when
using skinny extents} & {\small{}2013-06}\tabularnewline
\hline 
\end{tabular}
\vspace{-5pt}
\caption{\label{tab:Bug-reports}Bugs due to incorrect parsing of file system
formats.}
\vspace{-5pt}
\end{table*}

The rest of the paper is organized as follows. In Section~\ref{sec:Extended-Motivation}, we motivate the need for our approach. Section~\ref{sec:Approach} presents the core concepts that led to the design of the annotation language and the library API. Section~\ref{sec:File-System-Applications} describes the applications that we have implemented using the generated library. Section~\ref{sec:Implementation} describes the implementation of our system, and Section~\ref{sec:Evaluation} assesses the programming effort needed to annotate each file system and the performance of our applications. We present related work in Section~\ref{sec:Related_Work} and discuss our conclusions in Section~\ref{sec:Conclusion}.  For reference, Appendix~\ref{sec:Annotation_Language} shows our file system annotation language with examples of annotated structures for the Ext4, Btrfs and F2FS file systems. 


\section{Bugs in File-System Applications\label{sec:Extended-Motivation}}

We motivate this work by presenting various bugs caused by incorrect parsing of file-system metadata in storage applications (outlined in Table~\ref{tab:Bug-reports}). Some of these bugs cause crashes, while others may result in file system corruption. For each bug, we discuss the root cause.% and how our approach can help avoid it.

\begin{enumerate}[leftmargin=0.15in,itemsep=-0.25ex]

\item An extra memory allocation caused uninitialized bytes to be written to the boot jump field of Fat32 file systems during resizing. Since Windows depends on the correctness of this field, the bug rendered the file system unrecognizable by the operating system.

\item NTFS has a complex specification for the size of the MFT record. If the value is positive, it is interpreted as the number of clusters per record. Otherwise, the size of the record is $2^{|value|}$bytes (e.g., $-10$ would mean that the record size is 1024 bytes). The developers of ntfsprogs were unaware of this detail, and so the GParted partition editing tool would fail when attempting
to resize an NTFS partition.

\item In this version of e2fsck, the file system checker failed to detect corrupted directory entries if the size field of the entries was set to zero, which results in no repair being performed. Ironically, other programs, such as debugfs, ls, and the file system itself, can correctly detect the corruption.

\item Ext2/3/4 inodes contain union fields for storing operating system specific metadata. Unfortunately, a sanity check was omitted in e2fsck prior to accessing this field, causing erroneous repairs to be performed when the creator OS is not Linux. Consequently, the file system becomes corrupted for Hurd and possibly other operating systems.

\item A fuzzer that generates test cases to trigger different internal states of its target binary~\cite{AmericanFuzzyLop} was able to craft corrupted super blocks that would crash the Btrfsck tool. In response, Btrfs developers added 15 extra checks (for a total of 17 checks) to the super block parsing code.

%\item When the skinny metadata feature was added to Btrfs, the developers neglected to also patch Btrfsck, resulting in false error reports. This bug shows the difficulty of keeping all relevant applications up-to-date with changing file system formats.
\end{enumerate}

The common theme among all these bugs is that: 1) they are simple errors that occur because they require a detailed understanding of the file system format; 2) they can cause serious data loss or corruption; and 3) most of these bugs were fixed in less than 5 lines of code. Our domain-specific language allows generating libraries that can automatically sanitize file system metadata by checking various structural constraints before it is accessed in memory. In the presence of corrupted metadata, our libraries generate error codes, rather than crashing the tools, or propagating the corruption further.

\section{Approach\label{sec:Approach}}

The purpose of our file system annotation language is to enable robust interpretation of file system structures, in both offline and online contexts %, without requiring file-system specific code 
. Ideally, data structure types and their relationships could be extracted from file system source code. Although the C header files of a file system contain the structural definitions for various metadata types, they are incomplete descriptions of the file system format because information is often hidden within the file system code. Our annotations augment the C language, helping specify parts of a file system's format that cannot be easily expressed in C.


%-After a file system developer annotates their file system's data structures, we use a compiler to parse the annotated structures and to generate a library that provides file-system specific interpretation routines.  The library supports traversal and selective retrieval of metadata structures through type introspection. These facilities allow the application writer to create file-system specific policies that are applied to a subset of a file system's metadata. For example, the application may wish to operate on the directory entries of a file system. Instead of attempting to parse the entire file system and find all directory entries, which requires significant file-system specific code, a developer using Spiffy would perform selective traversal using type introspection to find and operate on directory entries. Since the directory entry format may not be the same across file systems, the application may still require file-system specific code, but this is essential to the application logic.

After a file system developer annotates their file system's data structures, we use a compiler to parse the annotated structures and to generate a library that provides file-system specific interpretation routines.  The library supports traversal and selective retrieval of metadata structures through type introspection. These facilities allow the application writer to create file-system specific policies that are applied to a subset of a file system's metadata. 

Application writers write code for each file-system specific structure they must process.
When Spiffy is instructed to follow a pointer, it determines the type of the pointed-to object and invokes the corresponding application code. The Spiffy library takes care of loading the requested object into memory, parsing it into fields, and validating that it meets any constraints given in the annotations. 

For example, the application may wish to enumerate directory entries in the file system (e.g. to build a snapshot of the file system's namespace). A developer using Spiffy would write a function for each file system's "directory entry" structure, which invokes a file-system agnostic function to record the file name, or any other pertinent information. In order to optimize file system traversal, the developer could also write decision making code for other objects, to determine whether to prune the traversal of some pointers within the file system (e.g. along unwanted directory entries, or along pointers to file system metadata that cannot contain directory entries.) In contrast, a manually written application requires file-system specific code to traverse the file system, loading and validating data structures, and parsing directory entries before actually invoking any application-specific code, namely, recording the filename in the directory entry.


Our annotation-based approach offers several advantages. First, it provides a concise and clear documentation of the file system's format.  Second, our generated libraries enable rapid-prototyping of file-system aware storage applications. The libraries provide a uniform API, easing the development of applications that work across file systems so that the programmer can focus on the logic and not the format of the file systems. Third, our approach requires minimal changes to the file system source code (the annotations are only in the C header files, and are backwards compatible with existing binary code), reducing the chance of introducing file system bugs. For example, differentiated storage services~\cite{Mesnier2011} was implemented by modifying the file system and the kernel's storage stack to enable I/O classification.  With our approach, this application can be implemented by using introspection at the block layer for an unmodified file system, or at the hypervisor for an existing virtual machine. Finally, file system formats are known to be stable over time, so there is minimal cost for maintaining annotations. When format changes do occur, the specifications need to be updated, as well as any file-system specific code which operate on the updated metadata structure.

\subsection{Designing Annotations\label{subsec:Annotation-Design}}

\begin{figure}[t]
\begin{lstlisting}
struct ext4_dir_entry { 
  __le32 inode;             /* Inode number */
  __le16 rec_len; /* Directory entry length */
  __u16  name_len;           /* Name length */
  char   name[EXT4_NAME_LEN];  /* File name */
};
\end{lstlisting}
\vspace{-10pt}
\caption{\label{fig:ext4_dirent}Ext4 directory entry structure definition.}
\end{figure}

\begin{figure}[t]
  \centering{}\includegraphics[width=1\columnwidth]{figures/f2fs_checkpoints}
\vspace{-20pt}
\caption{\label{fig:F2FS-checkpoint-packs}Each F2FS checkpoint pack contains a header followed by a variable number of orphan blocks.}
\vspace{-5pt}
\end{figure}

In this section, we describe the concepts that motivated the design of our annotation language for specifying the format of file system structures.

\paragraph{File System Pointers}

File system pointers connect the metadata structures in a file system, but they are not well specified in C data structure definitions, as explained in \ref{sec:Introduction}. 
%An in-memory pointer is a memory location that holds a virtual address of a data structure of a specific type. A file system pointer is similar, but file system addresses can be of different types. The most common type is a block type,
The difference between a file system pointer and an in-memory pointer is that the contents of an in-memory pointer are always interpreted as the in-memory address of the pointed-to data, but interpreting the address contained by a file system pointer may involve multiple layers of translation.
The most common type of file system pointer is a block pointer,
where the address maps to a physical block location that contains a contiguous data structure. However, file system structures may also be laid out discontiguously. For example, the journal of an Ext4 file system is a logically contiguous structure that can be stored on disk non-contiguously, as a file. Similarly, Btrfs maps logical addresses to physical addresses for supporting RAID configurations.


Our design incorporates this requirement by associating an \emph{address space} with each file system pointer. Each address space specifies a mapping of its addresses to physical locations. In the case of the Ext4 journal, we use the inode number, which uniquely identifies files in Unix file systems, as an address in the file address space.

Multiple pointers may refer to the same structure. For example, the block group descriptor tables in Ext4 refer to inode tables, which have inode structures in them. Similarly, directory entries have an inode number that also refer to these inode structures, as shown in Figure~\ref{fig:ext4_dirent}. We prefer to annotate the descriptor tables with a block type pointer because the inode table can then be viewed as a contiguous structure. Annotating the inode number is possible, but will require the annotation developer to implement an inode address space for mapping inode numbers to inode structures.

%
% (jsun): we have removed alias from the language
%
%When multiple pointer annotations lead to the same structure, the
%physical structure of the file system becomes a directed graph. We
%allow the annotation developer to specify pointers as \emph{aliases},
%so that the file system structures can be traversed in a preferred
%tree order, and alias pointers can be ignored or treated specially.

\paragraph{Cross-Structure Dependencies}

File system structures often depend on other structures. For example, the length of a directory entry's \texttt{name} in Ext4 is stored in a field called \texttt{name\_len}, as shown in Figure~\ref{fig:ext4_dirent}.  However, this data structure definition does not provide the linkage between the two fields.\footnote{This is especially confusing because \texttt{name} has a fixed size in the definition.} Structures may depend on fields in other structures as well. For example, several fields of the super block are frequently accessed to determine the block size, the features that are enabled in the file system, etc. To support these dependencies, we need to name these structures. For example, the expression \texttt{sb.s\_inode\_size} helps determine the size of an inode object, where \texttt{sb} is the name assigned to the super block.

The naming mechanism must ensure that a name refers to the correct structure. For example, the F2FS file system contains two checkpoint packs for ensuring file system consistency, as shown in Figure~\ref{fig:F2FS-checkpoint-packs}.  The number of orphan blocks in a F2FS checkpoint pack is determined by a field inside the checkpoint header. Our naming mechanism must ensure that when this field is accessed, it refers to the header structure in the correct checkpoint pack.

Spiffy uses a path-based name resolution mechanism, based on the observation that every file system structure is accessed along a path of pointers starting from the super block. In the simplest case, the automatic \texttt{self} variable is used to reference the fields of the same structure. Otherwise, a name lookup is performed in the reverse order of the path that was used to access the data structure. For example, in Figure~\ref{fig:F2FS-checkpoint-packs}, when we need to reference the checkpoint header (\texttt{cphdr} in the figure) while parsing the orphan block, the name resolution mechanism can unambiguously determine that it is referring to its parent checkpoint header.
%Our path-based naming approach works for both normal and alias pointers.
It also makes it easy to use reference counting to ensure that a referenced structure is valid in memory when it needs to be accessed.

\paragraph{Context-Sensitive Types}

File system metadata are frequently context-sensitive. A pointer may reference different types of metadata, or a structure may have optional fields, based on a field value. For example, the type of a journal block in Ext4 depends on a common field called \texttt{h\_blocktype}.  If its value is 3, then it is the journal super block that contains many additional fields that can be parsed. However, if its value is 2, then it is a commit block that contains no other fields. We need to be able to handle such context-sensitive structures and pointers.  We use a \emph{when} \emph{condition} clause, evaluated at runtime, to support such context-sensitive types.

\paragraph{Computed Fields}

Sometimes file systems compute a value from one or more fields and use it to locate structures. For example, the block group descriptor table in Ext4 is implicitly the block that immediately follows the super block. However, the exact address of the descriptor blocks depends on the block size, which is specified in the super block. We annotate this information as an implicit field of the super block that is computed at runtime. This approach allows the field to be dereferenced like a normal pointer, allowing traversal of the file system, but without requiring any changes to the underlying file-system format.

\paragraph{Metadata Granularity}

Existing file systems assume that the underlying storage media is a block device and access data in block units. Data structures can exist within such blocks or they can span contiguous physical blocks.  Some data structures that span blocks are read in their entirety.  For example, the Btrfs B-tree nodes are (by default) 16KB, or 4 blocks, and these blocks are read from disk together. In other cases, the data structure is read in portions. For example, an Ext4 inode table contains a group of inode blocks. The file system does not load the entire table in memory because it can be very large. Instead, it only loads the portions that are needed. For example, only one inode block needs to be fetched from disk to access an inode structure.

We define an \emph{access unit} for file system structures so that the compiler can generate efficient code for traversing the file system.  We call the unit of disk access a \emph{container}. The container size is typically the file system block size but it may span multiple blocks, as in the Btrfs example. A structure that is placed inside a container is called an \emph{object}. Finally, structures that span containers are called \emph{extents}. We load extents on demand, when their containers are accessed.

\paragraph{Constraint Checking}

The values of the metadata fields within or across different objects often have constraints. For example, an Ext4 extent header always begins with the magic number \texttt{0xF30A} to help detect corrupt blocks. Similarly, the \texttt{name\_len} field of an Ext4 directory entry should be less than the \texttt{rec\_len} field. Such constraints can be specified for each structure so that they can be checked to ensure correctness when parsing the structure.

The set of valid addresses for a metadata container may also have a \textit{placement constraint}. For example, F2FS NAT blocks can only be placed inside the NAT area, which is specified in the F2FS super block. By annotating the placement constraint of a metadata container, Spiffy can verify that the address assigned to newly allocated metadata is within the correct bounds before the metadata is persisted to disk.

\begin{table*}
\begin{centering}
\begin{small}
\begin{tabular}[t]{|c|l|l|}
\hline 
Base Class & Member Function & Description \tabularnewline
\hline
\multicolumn{3}{l}{Spiffy File System Library}\tabularnewline
\hline 
Entity & \texttt{int accept\_fields(Visitor \& v)} & allows \textit{v} to visit all fields of this object \tabularnewline
 & \texttt{int accept\_pointers(Visitor \& v)} & allows \textit{v} to visit all pointer fields of this object \tabularnewline
 & \texttt{int accept\_by\_type(int t, Visitor \& v)} & allows \textit{v} to visit all structures of type \textit{t} \tabularnewline 
\hline 
Pointer & \texttt{Entity * fetch()} & retrieves the pointed-to container from disk \tabularnewline
\hline
Container & \texttt{int save(bool alloc=true)} & \makecell[l]{serializes and then persists the container,
  may \\ assign a new address to the container
 }\tabularnewline
\hline
FileSystem & \texttt{FileSystem(IO \& io)} & instantiates a new file system object\tabularnewline
 & \texttt{Entity * fetch\_super()} & retrieves the super block from disk\tabularnewline
 & \texttt{Entity * create\_container(int type, Path \& p)} & creates a new container of metadata \textit{type}\tabularnewline
 & \makecell[l]{\texttt{Entity * parse\_by\_type(int type, Path \& p,} \\  \texttt{Address \& addr, const char * buf, size\_t len)}} & \makecell[l]{parses the buffer as metadata \textit{type}, using \\ \textit{p} to resolve cross structure dependencies \\ }\tabularnewline
\hline
\multicolumn{3}{l}{File System Developer}\tabularnewline
\hline 
IO & \texttt{int read(Address \& addr, char * \& buf)} & reads from an address space specified by \textit{addr}\tabularnewline
 & \texttt{int write(Address \& addr, const char * buf)} & writes to an address space specified by \textit{addr}\tabularnewline
 & \texttt{int alloc(Address \& addr, int type)} & allocates an on-disk address for metadata \textit{type}\tabularnewline 
\hline
\multicolumn{3}{l}{Application Programmer}\tabularnewline
\hline 
Visitor & \texttt{int visit(Entity * e)} & visits an entity and possibly process it\tabularnewline
\hline
\end{tabular}
\end{small}
\par\end{centering}
\vspace{-5pt}
\caption{\label{tab:library-api}Spiffy C++ Library API. }
\vspace{-5pt}
\end{table*}

\subsection{The Spiffy API\label{subsec:spiffy-api}}

Table~\ref{tab:library-api} shows a subset of the API for building Spiffy applications. The API consists of three sets of functions. The first set are automatically generated by Spiffy based on the annotated file system data structures. The second set need to be implemented by file system developers and are reusable across different applications. The last set are written by the application programmer for implementing application and file-system specific logic.

The Spiffy library uses the visitor pattern~\cite{gamma1995design} so that a programmer can customize the operations performed on each file system metadata type by implementing the \texttt{visit} function of the abstract base class \texttt{Visitor}.

%The \texttt{Entity} base class provides a common interface for all metadata, including objects (structures, vectors, fields), pointers, containers and extents. The \texttt{accept\_pointers} function invokes the \texttt{visit} function of an application-defined Visitor on each pointer within the entity.

The \texttt{Entity} base class provides a common interface for all metadata structures their fields. The \texttt{accept\_pointers} function invokes the \texttt{visit} function of an application-defined Visitor class on each pointer within the entity. \texttt{accept\_by\_type} allows visit of specified structure type reachable from the entity. Unlike the previous two versions of accept, \texttt{accept\_by\_type} will automatically follow pointers. Thus, for example, invoking \texttt{accept\_by\_type} on the super block with inode structure as the argument will result in all inodes in the file system being visited.

Every container (and extent) has an address associated with it that allows accessing the container from disk. Figure~\ref{fig:address-struct} shows the format of an address, consisting of an address space, an identifier and an offset within the address space, and the size of the container. The offset field is used when a container belongs to an extent.

\begin{figure}
\begin{lstlisting}
struct Address {
  int      aspc;   /* address space type */
  long 	   id;     /* id of the address */
  unsigned offset; /* offset from id */
  unsigned size;   /* size of object */
};
\end{lstlisting}
\vspace{-10pt}
\caption{\label{fig:address-struct}Address structure to locate container on disk.}
\vspace{-5pt}
\end{figure}

The \texttt{Pointer} class stores the address of a container (or an extent), and its \texttt{fetch} function reads the pointed-to container from disk. Figure~\ref{fig:fooptr-fetch} shows the generated code for the \texttt{fetch} function for a pointer to a container named \texttt{IBlock} (inode block). The file-system developer implements an \texttt{IO} class with a \texttt{read} function for each address space defined for the file system. When the \texttt{IBlock} is constructed, it invokes the constructors of its fields, thus creating all the objects (e.g., inodes) within the container. The constructors for inodes, in turn, invoke the constructors of block pointers in the inodes, which initialize a part of the address (address space, size and offset) of the block pointers based on the annotations. Then the container is parsed, which initializes the container fields in a nested manner, including setting the \texttt{id} component of the address of all the block pointers in the inodes contained in the \texttt{IBlock}.

The \texttt{Path} object is associated with every entity and contains the list of structures that are needed to resolve cross-structure dependencies during parsing or serializing the container. It is set up based on the sequence of constructor calls, with each constructor adding the current object to the path passed to it.

\begin{figure}
\begin{lstlisting}
Entity * IBlockPtr::fetch() {
  IBlock * ib;
  Address & addr = this->address;
  char * buf = new char[addr.size];
  this->fs.io.read(addr, buf);
  ib = new IBlock(this->fs, addr, this->path);
  ib->parse(buf, addr.size);
  return ib;
}
\end{lstlisting}
\vspace{-10pt}
\caption{\label{fig:fooptr-fetch}Example of a generated \texttt{fetch} function. \texttt{IBlockPtr} is a subclass of \texttt{Pointer}.}
\vspace{-5pt}
\end{figure}

%% During initialization, pointer fields set up the address space, size, and offset of its Address object ( see Figure~\ref{fig:address-struct}), which are known information provided by the annotations. During parsing, the pointer field gets the \texttt{id} associated with the address, which enables the actual fetch. This function works by creating a temporary buffer and use the \texttt{IO} object associated the file system to perform the actual disk access.

\begin{figure}
\begin{lstlisting}
int Container::save(bool alloc) {
  size_t len = this->address.size;
  char * buf = new char[len];
  this->serialize(buf, len);
  if (alloc)
    this->fs.io.alloc(this->address,
                      this->metadata_type);
  /* check placement constraint */
  this->fs.io.write(this->address, buf);
}
\end{lstlisting}
\vspace{-10pt}
\caption{\label{fig:object-save}Abbreviated version of the \texttt{save} function.}
\vspace{-5pt}
\end{figure}

Figure~\ref{fig:object-save} shows the \texttt{save} function for writing a container to disk. The function serializes the container by invoking nested serialization on its fields. Then, it invokes the \texttt{alloc} function for newly created metadata, or when existing metadata has to be reallocated (e.g., copy-on-write allocator). The allocator finds a new address for the container and updates any metadata that tracks allocation (e.g., the Ext4 block bitmap). If the address passes placement constraint checks, the buffer is written to disk.

%% For an application that creates new metadata, it should use the \texttt{create\_object} function. The \texttt{Path} object contains a list of cross structure references that the object may need during parsing. For example, when creating a directory block, the programmer needs to decide which inode the block belongs to, and appropriately assign it to the \texttt{Path} object.

The \texttt{create\_container} function constructs empty containers of a given type. The application developer can then fill the container with data and invoke \texttt{save} to allocate and write the newly created container to disk.
With online interpretation, the application already has the buffer containing the metadata, knows its type, and just needs to parse it. The \texttt{parse\_by\_type} factory function allows the programmer to bypass the \texttt{fetch} function and allows parsing of arbitrary buffers and constructing the corresponding containers, without the need for an \texttt{IO} object to read data from disk.

\subsection{Building Applications\label{subsec:Building-Applications}}

Figure~\ref{fig:fs-traversal-example} shows a sample application built using the Spiffy API. This application prints the type of each metadata block in an Ext4 file system in depth-first order. The \texttt{Ext4IO} class implements the block and the file address space, as described later in \ref{sec:Implementation}. The program starts by invoking \texttt{fetch\_super}, which fetches the super block from a known location on disk and parses it. Then it uses two mutually recursive visitors, \texttt{EntVisitor} and \texttt{PtrVisitor}, to traverse the file system.

The \texttt{EntVisitor::visit} function takes an entity as input, prints its name, and then invokes \texttt{accept\_pointers}, which calls the \texttt{PtrVisitor::visit} function for every pointer in the entity. The \texttt{PtrVisitor::visit} function invokes \texttt{fetch}, which fetches the pointed-to entity from disk, and invokes \texttt{EntVisitor::visit} on it.

\begin{figure}
\begin{lstlisting}
EntVisitor ev;
PtrVisitor pv;
int PtrVisitor::visit(Entity & e) {
  Entity * tmp = ((Pointer &)e).fetch();
  if (tmp != nullptr) {
    ev.visit(*tmp);
    tmp->destroy();
  }
  return 0;
}
int EntVisitor::visit(Entity & e) {
  cout << e.get_type().name << endl;
  return e.accept_pointers(pv);
}
void main(void) {
  Ext4IO io("/dev/sdb1");
  Ext4 fs(io);
  Entity * sup;
  if ((sup = fs.fetch_super()) != nullptr) {
    ev.visit(*sup);
    sup->destroy();
  }
}
\end{lstlisting}
\vspace{-10pt}
\caption{\label{fig:fs-traversal-example}Code for traversing and printing the types of all the metadata blocks in an Ext4 file system.}
\vspace{-5pt}
\end{figure}

Consider another application that shows file-system fragmentation by plotting a histogram of the size of free extents in the file system. This application will require file-system specific logic depending on how free space is represented, e.g., bitmaps for Ext4, and free space extents for Btrfs. This logic is implemented by modifying the code shown in Figure~\ref{fig:fs-traversal-example} with custom visit functions for just these structures, while the rest of the code in the application will be common across file systems.

\section{File System Applications\label{sec:File-System-Applications}}

We have written five file-system aware storage applications using the Spiffy framework: a dump tool, a free space reporting tool, a type-specific metadata corruptor, a file system conversion tool, and a prioritized block layer cache. The first four applications operate offline, while the last one is an online application. 

\vspace{-2pt}
\paragraph*{File System Dump Tool}

The file system dump tool parses all the metadata in a file system image and exports the result in an XML format, using file system traversal code similar to the example in Figure~\ref{fig:fs-traversal-example}. In addition to \texttt{process\_pointers}, the entity class provides a \texttt{process\_fields} method that allows iterating over all fields (not just pointer fields) of the class. The dump tool can be configured to prevent structures such as unallocated inode structures from being exported.

%Since some file system metadata may not be of interest, the XML writer provides APIs for ignoring fields or structures. For example, the Ext4 dump tool excludes any unallocated inode structures from being exported to XML. The tool currently supports Ext3/4, Btrfs, and F2FS.

%% For each file system, a main function initializes an instance of the file system class, the address space translation module, and the XML writer, and then runs file system traversal code similar to the example shown in Figure~\ref{fig:fs-traversal-example}.

\vspace{-2pt}
\paragraph*{Type-Specific Corruption Tool}

This tool is a variant of the dump tool that injects file-system corruption in a type-specific manner~\cite{bairavasundaram2008analyzing}, allowing us to test the robustness of file systems and their tools. When we decide to corrupt a field, we cannot simply modify its in-memory value, since serialization is type-safe. For example, the serializer will refuse to serialize a corrupted value that violates its type constraints. Instead, corruption is performed after a block is serialized but before it is written.

% For example, if we increase the length of a directory entry, the serialized directory block will shift the rest of the entries. 

\vspace{-2pt}
\paragraph*{Free Space Tool}

This tool shows file-system fragmentation by plotting a histogram of the size of free extents. The tool retrieves the metadata structures that store free space information and processes them (e.g., block bitmaps for Ext4, extent items for Btrfs, and segment information table (SIT) for F2FS). This logic is implemented using \texttt{process\_by\_type} (see Table~\ref{tab:library-api}) and a custom visit function that processes all the retrieved metadata structures. Code to traverse the file system and parse intermediate structures is provided by our library. 
%The rest of the application code is common across file systems.

%% The free space tool displays the offsets and sizes of all free space extents in a file system image. A histogram of the free extent counts is also displayed, with power of two bin sizes.

%% The tool supports Ext4, Btrfs, and F2FS. The file system specific code generates a bitmap of free blocks in the image. The Ext4 implementation visits the block bitmaps within the group descriptor table, using \texttt{process\_by\_type}. The Btrfs implementation populates a bitmap using the extent start and size values from the extent items and metadata items (which include system extents) in the extent tree. The F2FS implementation generates a bitmap for the newest valid checkpoint version. It visits by type and records the SIT entries in the SIT extent. It also visits by type the sit entries in the checkpoint journal, recording them with higher precedence in the bitmap. Finally, the generic code converts the bitmap generated by the file-system specific code to an extent-based representation, and displays it as a list and histogram. 

%This additional complexity is the cause for the greater line count of the F2FS specific functionality.  The generic code contains a routine for detecting the file system type of an image, which executes the correct file system specific code.

\vspace{-2pt}
\paragraph*{File System Conversion Tool}

Converting an existing file system into a file system of another type is a time-consuming process, involving copying files to another disk, reformatting the disk, and then copying the files back to the new file system. In-place file system conversion that updates file system metadata without moving most file data can speed up the conversion dramatically. While some such conversion tools exist,\footnote{The {convert} utility converts FAT32 to NTFS~\cite{fat-to-ntfs}, and updating to iOS 10.3 upgrades the file system from HFS+ to APFS~\cite{apfs-upgrade}} they are hard to implement correctly and not generally available.

% such techniques are also used for upgrading a file system to a newer version, s

We have designed an in-place file system conversion tool using the Spiffy framework. Such a conversion tool requires detailed knowledge of the source and the destination file systems, and is thus a challenging application for our approach. In-place conversion involves several steps. First, the file and directory related metadata, such as inodes, extent mappings, and directory entries of the source file system, are parsed into a standard format. Second, the free space in the source file system is tracked. Third, if any source file data occupies blocks that are statically allocated in the destination file system, then those blocks are reallocated to the free space, and the conversion aborted if sufficient free space is not available. Finally, the metadata for the destination file system is created and written to disk. In our current tool, a power failure during the last step would corrupt the source file
system. We plan to add failure atomicity in the future.

Our tool currently converts extent-based Ext4 file systems to log-structured F2FS file systems. The source file system is read using a custom set of visitors that efficiently traverse the file system and create in-memory copies of relevant metadata. For example, unused block groups can be skipped while processing block group descriptors. Next, we generate the free space list by reusing components from the free space tool, and then removing F2FS's static metadata area from the list. Then, Ext4 extents in the F2FS metadata area are relocated to the free space with their mappings updated. Finally, F2FS metadata is created from the in-memory copies and written to disk, which involves allocation and pointer management, requiring significant file-system-specific logic.

Fortunately, various pieces of the code can be reused for different combinations of source and destination file system when adapting new file systems. As an example, only the code to copy Btrfs metadata from an existing file system and to list its free space is required to support the conversion from Btrfs to F2FS, since the in-memory data structures are generic across file systems that support VFS. If the file system does not support VFS, suitable default values can be used, which would be helpful for upgrading from a legacy file system such as FAT32.

%
% (jsun): REMOVED for FAST submission
%
%\begin{figure}[t]
%\begin{lstlisting}
%class Ext4DirEntry : public Entity {
%  Ext4DirEntry() :
%    inode("inode", "__le32"),
%    ...
%    name("name", "char []", *this) {}
%public:
%  Integer<__le32> inode;
%  Integer<__le16> rec_len;
%  Integer<__u16>  name_len;        
%  Vector<char>    name;
%  ...
%};
%\end{lstlisting}
%\vspace{-12pt}
%\caption{\label{fig:ext3_dirent_library_wrapper}Wrapper class for Ext4 directory
%entry.}
%\vspace{-8pt}
%\end{figure}

\vspace{-2pt}
\paragraph{Prioritized Block Layer Cache}

We have implemented a file-system aware block layer cache based on Bcache \cite{Bcache2016}. Our cache preferentially caches the files of certain priority users, identified by the \texttt{uid} of the file. This caching policy can dramatically improve workload performance by improving the cache hit rate for prioritized workloads, as shown in previous work~\cite{sw-defined-caching-2015}.  Bcache uses an LRU replacement policy; in our implementation, blocks belonging to priority users are given a second chance and are only evicted if they return to the head of the LRU list without being referenced.

%
% (jsun): parse_by_type later described in runtime interpretation, and is not
% specific to Ext4 
%
% we use the \texttt{parse\_by\_type()} function shown in Table~\ref{tab:library-api} and ...
%
We use a runtime interpretation module, described in more detail in Section~\ref{sec:Implementation}, to identify metadata blocks at the block layer without any modifications to the file system. We track the data extents that belong to file inodes containing the \texttt{uid} of a priority user, so that we can preferentially cache these extents. For Ext4, we use custom visit functions to parse inodes and determine the priority extent nodes. Similarly, we parse the priority extent nodes to determine the priority extent leaves, which contain the priority data extents.

%% For Ext4, we process the extent tree inside an inode with the \texttt{uid}. If the extent tree has multiple levels, we separately track the block address for the extent nodes and leaves so that when we later process an extent leaf, we can add the data extents in the leaf to our list of priority extents.

For Btrfs, the inodes and their file extent items may not be placed close together (e.g., within the same B-tree leaf block), and so parsing an inode object will not provide information about its extents. Fortunately, the key of a file extent item is its associated inode number, making it easy to track the file extents of priority users.

%% Therefore, whenever we come across a file inode belonging to the priority user, we add the inode number to another list so thatwe can identify file extent item that contains priority data extents.

%% The dump tool and the differentiated block layer cache requiring minimal file-system specific code, while the file system conversion tool require knowledge of both the source and destination file system to convert metadata from one format to another.

\section{Implementation\label{sec:Implementation}}

We implemented a compiler that parses Spiffy annotations. (These annotations are described in Table~\ref{tab:annotation_language} of Appendix~\ref{sec:Annotation_Language})
%
%% REMOVED for the FAST submission - Ashvin
%
%% The compiler uses Python Lex-Yacc (PLY 3.4)~\cite{pythonlexyacc} as its parser generator and lexical analyzer. The grammar, written in Yacc, is based on the ANSI C grammar. The compiler is invoked with a set of C header files containing the annotated data structures (e.g., \texttt{spiffy --name Ext4 ext4\_fs.h}). It parses the annotations and the annotated structures, and ignores the rest of the source code. We verify that the boolean and integer expressions used in annotations are syntactically correct by attempting to compile the expressions using a C++ compiler.
%
The compiler generates the file system's internal representation in a symbol table, containing the definitions of all the file system metadata, their annotations, their fields (including type and symbolic name), and each of their field's annotations. Next, it detects errors such as duplicate declarations or missing required arguments. Finally, the symbol table and compiler options are exported for use by the compiler's backend. 

Spiffy's backend generates a file-system specific metadata library using Jinja2~\cite{ronacher2011jinja2}.
%%a templating engine that is typically used for generating dynamic HTML content. The code generator works by processing template filters and tags in the source template files, and the output of the compiler is a pair of C++11 source and header files that can be used by applications. Our system can be compiled as either a user space library or a Linux kernel module. 
The library can be compiled as either a user space library or as part of a Linux kernel module. We linked our generated library into the Linux kernel by porting some C++ standard containers and integrating the GNU g++ compiler into the kernel build process with some minor changes.

Every annotated structure is wrapped in a class that implements the \texttt{Entity} interface. Figure~\ref{fig:ext3_dirent_library_wrapper} shows an example structure for the Ext4 directory entry. The \texttt{name} field is initialized with its name and type for introspection, and also a reference to the structure so that it can reference \texttt{self} during parsing. We make each of the fields publicly visible by using the cast and assignment operators in the field's template class. Application programmers can thus access these fields as if they were accessing the actual C structure.

%% The simple file-system traversal application shown in Figure~\ref{fig:fs-traversal-example} uses these entity structures.

The generated library performs various types of error-checking operations. For example, the parsing of offset fields ensures that objects do not cross container boundaries, and that all variable-sized structures fit within their containers. These checks are essential if an application aims to handle file system corruption. When parsing does fail, an error code is propagated to the caller of the \texttt{parse} or \texttt{serialize} function. Spiffy guarantees that as long as all type-safety checks pass, then parsing and serialization will not cause a crash.

%% REMOVED for the FAST submission - Ashvin

%% Our path-based name resolution mechanism resolves a name in the reverse order of the path from the super block. For example, suppose the path is $A\rightarrow B_{1}\rightarrow C\rightarrow B_{2}\rightarrow D$, where each symbol is a unique structure type, and $B_{1}$and $B_{2}$ are separate instances of the same type. Structure $C$ would resolve \emph{B} as $B_{1}$, but \emph{D} would resolve \emph{B} as $B_{2}$, and not see $B_{1}$. However, it can use $C.B$ to access $B_{1}$. This mechanism is implemented by associating a path object and a parent entity with each entity. After a pointer is used to fetch and parse an entity, its path object is created by performing a shallow copy of the parent's path object, and appending a pointer to the current entity. The shallow copy increments entity reference counts, ensuring that the names in the path can be referenced correctly. When a name is not specified for a structure (in the \texttt{FSSTRUCT} annotation), the corresponding entity is not added to the path.

Currently, the fetch function always reads data from storage because we have not implemented an entity cache. This doesn't affect a tree traversal in which each entity is read once, but if a structure can be accessed using multiple paths, then it would be read multiple times. 

\paragraph*{Address Spaces}

We require annotation developers to implement the \texttt{IO} interface shown in Table~\ref{tab:library-api}. The read method maps a pointer address in an address space to a physical location on disk, and then reads a container of a given size, specified by \texttt{addr.size}, into the buffer \texttt{buf}.  \texttt{addr.offset} is used to read a container within an extent. We require a byte address space implementation so that the super block can be fetched at a fixed byte offset on disk. The super block usually contains the block size, which enables a block address space implementation.

The Ext4 file address space implementation for the \texttt{Ext4IO} class (see Figure~\ref{fig:fs-traversal-example}) requires fetching the file contents associated with an inode number. It requires reading the corresponding inode structure, converting the size and offset arguments into a list of physical block numbers, fetching these blocks into memory, and combining the blocks together. For Btrfs, we currently support the RAID address space for a single device, which only allows metadata mirroring (RAID-1). For F2FS, we support the NID address space, which maps a NID (node id) to a node block. The implementation involves a lookup to see if a valid mapping entry is the journal. If not, the mapping is obtained from the node address table.

\paragraph*{Runtime Interpretation}

Offline Spiffy applications use variants of the file-system traversal algorithm shown in Figure~\ref{fig:fs-traversal-example}. Spiffy also supports online file-system aware storage applications.  To do so, we programmed a module to perform file system interpretation at the block layer of the Linux kernel using the generated libraries. This class of file system application is typically difficult to write and error prone, since manual parsing code is needed for each block type. However, our implementation only requires a small amount of bootstrap code to support any annotated file system. The rest of the code is file-system independent.

In offline applications, the fetch function reads data from disk and parses the structure. The type of the structure is known from the pointer that is passed to the fetch function. In contrast, for online interpretation, the file system performs the read, and the module only parse the block that is read by calling \texttt{FileSystem::parse\_by\_type()}. However, it needs to know the type of the block before parsing is possible. Our runtime interpretation depends on the fact that a pointer to a metadata block must be read before the pointed-to block is read. When a pointer is encountered during the parsing a block, the module tracks the type of the pointed-to block. As a result, when the pointed-to block is read, its type is known.

Our module exports several functions, including \texttt{interpret\_read} and \texttt{interpret\_write}, that need to be placed in the I/O path to perform runtime interpretation. The module maintain a mapping between block numbers and their types. After intercepting a completed read request, it checks whether a mapping exists in the mapping table, and if so, it is a metadata block and it gets parsed. Next, \texttt{process\_pointers} is invoked with a visitor that adds (or updates) all the pointers that are found in the block into the mapping table.

When the I/O operation is a write, the module needs to determine the type of the written block. A statically allocated block can be immediately parsed because its type will not change. For example, most metadata blocks in Ext4 are statically allocated. However, in Btrfs, the super block is the only statically allocated metadata block. For dynamically allocated blocks, the block must first be labeled as unknown and its contents cached, since its type may either be unknown or have changed. Interpretation for this block is deferred until it is referenced by a block that is subsequently accessed (either read or written), and whose type is known. At that point, the module will interpret all unknown blocks that are referenced.

Since most dynamically-typed blocks are data blocks, they should be discarded immediately to reduce memory overhead. For the Btrfs file system, this is relatively easy because metadata blocks are self-identifying.  For Ext4, these blocks need to be temporarily buffered until they can be interpreted. However, we use a heuristic for Ext4 to quickly identify dynamically-typed blocks that are definitely not metadata, to reduce the memory overhead of deferred interpretation. The block is first parsed as if it were a dynamically allocated block (e.g., a directory block or extent metadata block), and if the parsing results in an error, then the block is assumed to be data and discarded. This heuristic could be used in other file systems as well because most file systems have a small number of dynamically allocated metadata block types, or their blocks are self-identifying.

%Our runtime interpretation system is currently used to read file system blocks. Supporting applications that write blocks, such as IO shepherding~\cite{Gunawi07}, will require an IO manager that can make its own IO requests, which is outside the scope of what Spiffy provides.

The module currently relies on the file system to issue \texttt{trim} operations to detect deallocation of blocks so that stale entries can be removed from the mapping table. Since file systems do not guarantee correct implementation of \texttt{trim}, the module additionally flushes out entries for dynamically allocated blocks that have not been accessed recently. This works for a caching application, but may lead to mis-classification for other runtime applications. Accurate classification can be implemented by keeping the previous versions of blocks and comparing the versions at transaction commit time. However, it comes with a higher memory overhead~\cite{Fryer2012b}.

%% This is because of write ordering. If the pointer to a block is written before the block itself is written, the written block will be marked as "unknown" until a subsequent pointer to it is written. However, since the pointer is written first, the block can never be classified with our implementation. During checkpointing, write ordering is not guaranteed. This is one of the issues with not interpreting the journal but instead the final location of metadata.

\subsection{Limitation\label{subsec:spiffy_limitation}}

Spiffy does not explicitly support concurrency and the application must deal
with synchronization manually if it is multi-threaded. We currently do 
not support online modification of file system metadata.
There are some optimization
that Spiffy has yet to implement, such as support for readahead while iterating 
through contiguous extents.

The correctness of Spiffy applications currently depend on the correct and
complete annotations. Therefore, when file system format changes, Spiffy
applications will also need to update all of the file-system specific code
that is affected by the change in semantics, as well as updating the annotations 
to reflect its current format. While this appears to involve some work, it will be
less than updating a set of manually-written tools since robust parsing
and serialization routines are already provided by the library.

Our annotation strategy makes two assumptions about file systems. We assume that the interpretation of a piece of file system metadata does not depend on the interpretation of children reached by traversing that same metadata. We also assume that any context-sensitive interpretations can be performed by a function or expression free of side effects.



\section{Evaluation\label{sec:Evaluation}}

In this section, we discuss the effort required to annotate the structures
of existing file systems, the effort required to write Spiffy applications, and then evaluate the performance of our file-system conversion tool and the file-system aware block-layer caching mechanism.

\subsection{Annotation Effort\label{subsec:Annotation-Effort}}

\begin{table}
\begin{centering}
\begin{tabular}{|c|c|c|c|}
\hline 
File System & Line Count & Annotated & Structures\tabularnewline
\hline 
\hline 
Ext4 & 491 & 113 & 15+10+4\tabularnewline
\hline 
Btrfs & 556 & 151 & 27+4+1\tabularnewline
\hline 
F2FS & 462 & 127 & 14+16+5\tabularnewline
\hline 
\end{tabular}
\par\end{centering}
\vspace{-5pt}
\caption{\label{tab:loc-struct}File system
structure annotation effort.}
\vspace{-5pt}
\end{table}

Table~\ref{tab:loc-struct} shows the effort required to correctly annotate the Ext4, Btrfs and F2FS file systems. The second column shows the number of lines of code of existing on-disk data structures in these file systems. The lines of code count was obtained using \texttt{cloc}~\cite{danial2009cloc} to eliminate comments and empty lines. The third column shows the number of annotation lines. This number is less than one-third of the total line count for all the file systems.

The last column is listed as $A+B+C$, with $A$ showing no modification to the data structure (other than adding annotations), $B$ showing the number of data structures that were added,\footnote{We consider the vector type to be an annotation and not a structure for this calculation.} and $C$ showing the number of data structures that needed to be modified. Structure declarations needed to be added or modified for three reasons:
%
\begin{enumerate}[leftmargin=0.15in,itemsep=-0.5ex]
\item We break down structures that benefit from being declared as conditionally
inherited types. For example, \texttt{btrfs\_file\_extent\_item}
is split into two parts: the header and an optional footer, depending on
whether it contains inline data or extent information for data.
\item Simple structures such as Ext4 extent metadata blocks, are not declared in
the original source code. However, for annotation purposes, they need
to be explicitly declared. All of the added structures in Ext4 belong
to this category.
\item Some data structures with a complex or backward-compatible format require 
modifications to enable proper annotation. For example, Ext4 inode retains its Ext3 definition in the official header file even though the \texttt{i\_block} field now contains extent tree information rather than block pointers. We redefined 
the Ext4 inode structure and replaced \texttt{i\_block} with the extent header
followed by four extent entries.
\end{enumerate}

\subsection{Application Developer Effort\label{subsec:Developer-Effort}}

In this section, we evaluate the effort required to develop Spiffy applications. 

\noindent\textbf{Dump Tool:} The file system dump tool is a simple tool in which the file-system independent XML writer module is written in 482 lines of C++ code, and the main function for each file system is written in 30 to 60 lines of code, depending on the number of structures the programmer wants to skip. The dump tool is helpful for debugging issues with real file systems, and to verify the correctness of the annotations. In particular, an expert can verify that the annotations are correct when the output of the dump tool matches the expected contents of the file system. Therefore, this tool has become an integral tool in our development process.

\noindent\textbf{Free Space Tool:} The file system free space tool is written using 271 lines of file system independent C++ code, along with additional code used to retrieve free space information from each file system. In addition, 76 lines are Ext4 specific, 77 lines are Btrfs specific, and 194 lines are F2FS specific. F2FS requires more code due to the complex format of its block allocation information.

\noindent\textbf{Conversion Tool:} The Spiffy file system conversion tool framework is written in 504 lines of code. The code for reading Ext4 takes 218 lines, the code to convert to the F2FS file system requires 1760 lines, and the file-system developer code for F2FS, which is reused in other applications such as the dump tool, consists of 383 lines. We also wrote a manual converter tool that uses the \texttt{libext2fs}~\cite{tso-e2fsprogs} library to copy Ext4 metadata from the source file system, and manually writes raw data to create an F2FS file system. The manual converter has 223 lines of Ext4 code, and 2260 lines for the F2FS code. In this case, the two converters have similar number of lines of code, but the Spiffy converter has several other benefits. On the source side, the manual converter takes advantage of the \texttt{libext2fs} library. Changing this code for a different file system would require significant changes, and would require much more code for a file system such as ZFS that does not have a similar user-level library. On the destination side, the main reason that the Spiffy converter requires many file-system specific lines of code is that each newly created object needs to be initialized, and the initialization has to be performed manually. However, Spiffy uses the \texttt{create\_container} and \texttt{save} functions to create and serialize objects in a type-safe manner, and checks constraints on objects, while the manual converter writes raw data, which is error-prone, leading to the types of bugs discussed in \ref{sec:Extended-Motivation}.

\noindent\textbf{Prioritized Cache:} The original Bcache code consisted of 10518 lines of code. To implement prioritized caching we added 289 lines to this code, which invokes our generic runtime metadata interpretation framework, consisting of 2158 lines of code. This framework provides hooks to specify file-system specific policies. Our Ext4-specific policy requires 111 lines of code, and the Btrfs-specific policy requires 134 lines of code. Currently, we have not implemented prioritized caching for F2FS, which would require tracking NAT entries, similar to how we track inode numbers for Btrfs to find their file extents.

\subsection{File System Conversion Performance\label{subsec:fsconvert_performance}}

\begin{table*}
\begin{centering}
\begin{tabular}{|c|c|c|c|c|}
\hline 
\# of files & 20000 & 5000 & 1000 & 100 \tabularnewline
%\hline
%File set size & 16.12GB & 16.27GB & 16.13GB & 16.27GB \tabularnewline
\hline 
\hline 
Copy Convertor & $188.17 \pm 3.65$s & $190.28 \pm 2.15$s & $192.74 \pm 2.28$s & $195.11 \pm 0.18$s \tabularnewline
\hline 
Manual Converter & $6.55\pm 0.53$s & $3.46 \pm 0.17$s & $3.29 \pm 0.11$s & $3.25 \pm 0.11$s \tabularnewline
\hline 
Spiffy Converter & $7.03 \pm 0.2$s & $4.01 \pm 0.09$s & $3.84 \pm 0.03$s & $3.71 \pm 0.13$s\tabularnewline
\hline 
\end{tabular}
\par\end{centering}
\vspace{-5pt}
\caption{\label{tab:fsconvert-result}Time required for each technique to convert 
from Ext4 to F2FS for different number of files.}
\vspace{-5pt}
\end{table*}

We compare the time it takes to perform copy-based conversion, versus using the Spiffy-based and the manually written in-place file-system conversion tools. The results are shown in Table \ref{tab:fsconvert-result}. The experiments are run on an Intel 510 Series SATA SSD.  We create the file set using Filebench 1.5-a3~\cite{wilson2008new} in an Ext4 partition on the SSD, and then convert the partition to F2FS. The 20K file set uses the \texttt{msnfs} file size distribution with the largest file size up to 1GB. The rest of the file sets have progressively fewer small files. All file sets have a total size of 16GB. For the copy converter, we run \texttt{tar -aR} at the root of the SSD partition and save the tar file on a separate local disk. We then reformat the SSD partition and extract the file set back into the partition.

The copy converter requires transferring two full copies of the file set, and so it takes 30 to 50 times longer than using the conversion tools, which only need to move data blocks out of F2FS's static metadata area and then create the corresponding F2FS metadata. Both conversion tools take longer time with  larger filesets since they need to handle the conversion of more file system metadata. The library-assisted conversion tool performs reasonably compared to its manually-written counterpart, with at most a 16.7\% overhead for the added type-safety protection that the library offers. This shows the feasibility of using the library for general use when working with file system metadata. 

%
% Jack: should we talk about why mount-and-copy does slightly better with more small files?
%

\subsection{Prioritized Cache Performance\label{subsec:pcache_performance}}

We measure the performance of our prioritized block layer cache (see \ref{sec:File-System-Applications}), and compare it against LRU caching with one or two instances of the same workload.

Our experimental setup includes a client machine connected to a storage server over a 10Gb Ethernet using the iSCSI protocol. The storage server runs Linux 3.11.2 and has 4 Intel Processor E7-4830 CPUs for a total of 32 cores, 256GB of memory and a software RAID-6 volume consisting of 13 Hitachi HDS721010 SATA2 7200 RPM disks. The client machine runs Linux 4.4.0 with Intel Processor E5-2650, and an Intel 510 Series SATA SSD that is used for client-side caching. To mimic the memory-to-cache ratio of real-world storage servers, we limit the memory on the client to 4GB and use 8GB of the SSD for write-back caching. The RAID partition is formatted with either the Ext4 or Btrfs file system and is used as the primary storage device. To avoid any scheduling related effects, the NOOP I/O scheduler is used in all cases for both the caching and primary device.

We use a pair of identical Filebench fileserver workloads to simulate a shared hosting scenario with two users where one requires higher storage performance than the other.
%We use a pair of fileservers, running on the same machine,
%with one having cache priority over the other.
We generate a total file set size of 8GB with an average file size of 128KB, for each workload. The fileserver personality performs a series of create, write, append, read and delete of random files throughout the experiment. Filebench is set to report performance metrics every 60 seconds over a period of 90 minutes. However, since each experiment starts with an empty cache, the performance initially fluctuates. Thus, we present the average of the results from the last 60 minutes of the experiment, after performance stabilizes.

\begin{figure}
\begin{centering}
\includegraphics[width=1\columnwidth]{figures/caching-result-stacked}
\par\end{centering}
\vspace{-5pt}
\caption{\label{fig:Performance-of-differentiated}Throughput of prioritized caching over LRU caching with one or two file servers for Ext4 and Btrfs.}
\end{figure}

Figure \ref{fig:Performance-of-differentiated} shows the average throughput for each of the experiments in operations per second.  The error bars show 95\% confidence intervals.  First, we establish the baseline performance of a single fileserver instance running alone, which has a cache hit ratio of 64\% and 54\% for Ext4 and Btrfs, respectively. Next, we run two instances of fileserver to observe the effect of cache contention. We see a drastic reduction in cache hit ratio to 23\% and 24\% for Ext4 and Btrfs, respectively.  Both fileservers have similar performance, which is between 2.3x and 2.7x less than when running alone. When we apply preferential caching to the files used by fileserver A, however, its throughput improves by 60\% over non-prioritized LRU caching when running concurrently with fileserver B, with the overall cache hit ratio improving to 46\% and 53\% for Ext4 and Btrfs, respectively.  Interestingly, prioritized caching also improves the aggregate throughput of the system by 14\% to 22\%. Giving priority to one of the two jobs implicitly reduces cache contention. These results show that storage applications using our generated library can provide reasonable performance improvements without the need to change the file system code.

\section{Related Work\label{sec:Related_Work}}

A large body of work has focused on storage-layer applications that perform file-system specific processing for improving performance or reliability. Semantically-smart disks~\cite{Sivathanu03} used probing to gather detailed knowledge of file system behavior, allowing functionality or performance to be enhanced transparently at the block layer. The probing was designed for Ext4-like file systems and would likely require changes for copy-on-write and log-structured file systems. Spiffy annotations avoid the need for probing, helping provide accurate block type information based on runtime interpretation.

I/O shepherding~\cite{Gunawi07} improves reliability by using file structure information to implement checksumming and replication. Block type information is provided to the storage layer I/O shepherd by modifying the file system and the buffer-cache code. Our approach enables I/O shepherding without requiring these changes. {\color{red}Also, unlike I/O shepherding, Spiffy allows interpreting block contents, enabling more powerful policies, such as caching the files of specific users.}

A type-safe disk extends the disk interface by exposing primitives for block allocation and pointer relationships~\cite{Sivathanu06}, which helps enforce invariants such as preventing access to unallocated blocks, but this interface requires extensive file system modifications. We believe that our runtime interpretation approach allows enforcing such type-safety invariants for existing file systems.
%, or allowing accesses to a Block X only after a Block Y that has a valid reference to Block X has been accessed.

Serialization of structured data has been explored through interface languages such as ASN.1~\cite{steedman1993abstract} and Protocol Buffers~\cite{varda2008protocol}, which allow programmers to define their data structures so that marshaling routines can be generated for them. However, the binary serialization format for the structures is specified by the protocol and not under the control of the programmer. As a result, these languages cannot be used to interpret the existing binary format of a file system.

Data description languages such as Hammer~\cite{pattersonhammer} and PADS~\cite{fisher2011pads} allow fine-grained byte-level data formats to be specified. However, they have limited support for non-sequential processing, and thus their parsers cannot interpret file system I/O, where a graph traversal is required rather than a sequential scan. Furthermore, with online interpretation, this traversal is performed on a small part of the graph, and not on the entire data.

Nail~\cite{bangert2014nail} shares many goals with our work. Its grammar provides the ability to specify arbitrarily computed fields. It also supports non-linear parsing, but its scope is limited to a
single packet or file, and so it does not support references to external objects. Our annotation language overcomes this limitation by explicitly annotating pointers, which defines how file system metadata reference each other. We also provide support for address spaces, so that address values can be mapped to user-specified physical locations on disk.

Several projects have explored C extensions for expressing additional semantic information~\cite{Necula2002,zhou2006safedrive,torvalds2007sparse}. CCured~\cite{Necula2002} enables type and memory safety, and the Deputy Type System~\cite{zhou2006safedrive} prevents out-of-bound array errors. Both projects annotate source code, perform static analysis, and add runtime checks, but they are designed for in-memory structures.

%
% (jsun): REMOVED for FAST submission
%
%Symbolic execution~\cite{cadar2008exe}, model checking~\cite{yang2006using} and fuzz-testing~\cite{AmericanFuzzyLop} have been used to find file system bugs. 
%These approaches involve discovering bugs by either using static analysis or by crafting inputs that would trigger faulty corner cases. 
%In contrast, we proactively avoid bugs by adding type safety checks in the generated parsing and serializing routines. Nonetheless, these technique would be helpful for detecting annotation errors that may be encountered during the development phase.

Formal specification approaches for file systems~\cite{amani2012towards,chen2015using}
require building a new file system from scratch, while our work focuses on building tools for existing file systems. Chen et al.~\cite{chen2015using} use logical address spaces as abstractions
for writing higher-level file system specifications.
%without needing to handle the details of the underlying implementation that has already been proved. 
This idea inspired our use of an address space type for specifying pointers. Another method for specifying pointers is by defining paths that enable traversing the metadata tree to locate a metadata object, such as finding the inode structure from an inode number~\cite{hesselink2009formalizing,gardner2014local}. These approaches focus on the correctness of file-system operations at the virtual file system layer, whereas our goal is to specify the physical structures of file systems.

\vspace{-1ex}


%\section{Future Work\label{sec:Future_Work}}

%In the current version of our file system conversion tool, a power failure while overwriting the metadata of the source file system would permanently corrupt it. We plan to add failure atomicity to the tool in the next iteration.

%Spiffy currently has no semantic understanding of file system metadata, which requires application developers to write file-system specific code.  However, there are common concepts between file system such as the notion of files and directories, and the allocation of blocks, even though each file system may choose different format to represent these concepts.  To bridge the semantic gap, we are working on a new interface which enables writing truly file-system agnostic applications by identifying these commonalities.  Application developers can use the interface to perform tasks such as allocating a new block, and copying content to it, all without knowing the underlying file system implementation.

\section{Conclusion\label{sec:Conclusion}}

Spiffy is an annotation language for specifying the on-disk file system data structures. File system developers annotate their data structures using Spiffy, which enables generating a library that allows parsing and traversing file system data structures correctly.

%We have demonstrated the simplicity and expressiveness of the language, and 
We have shown the generality of our approach by annotating three vastly different file systems. The annotated file system code serves as a detailed documentation for the metadata structures and the relationships between them. File-system aware storage applications can use the Spiffy libraries to improve their resilience against parsing bugs, and to reduce the overall programming effort needed for supporting file-system specific logic in these applications. Our evaluation suggests that applications using the generated libraries perform reasonably well. We believe our approach will enable interesting applications that require an understanding of storage structures.

\section*{Acknowledgements}
We thank the anonymous reviewers and our shepherd, Andr\'e Brinkmann, for 
their valuable feedback. This work was supported by NSERC.

{\footnotesize \bibliographystyle{acm}
\bibliography{bibliography}}
%\appendix

\begin{table*}
\begin{centering}
\begin{tabular}[t]{l>{\raggedright}p{0.2\textwidth}l>{\raggedright}p{0.41\textwidth}}
\multicolumn{1}{c}{\textbf{Keyword}} & \multicolumn{1}{c}{\textbf{Description}} & \multicolumn{1}{c}{\textbf{Arguments}} & \multicolumn{1}{c}{\textbf{Meaning}}\tabularnewline
\hline 
\texttt{FSSTRUCT} & File system structure & \texttt{name=IDENT} & Name of the structure for cross referencing\tabularnewline
 &  & \texttt{unit=UNIT} & Can be\texttt{ object}, \texttt{container}, or \texttt{extent}\tabularnewline
\texttt{FSSUPER} & File system super block & \texttt{base=TYPE,when=BOOL} & Structure inherits from base when condition is true\tabularnewline
 &  & \texttt{size=INT} & Size of the structure\tabularnewline
 &  & \texttt{location=INT} & Location of the super block or placement constraint\tabularnewline
\hline 
\texttt{POINTER} & \multirow{2}{0.2\textwidth}{Field is a pointer to a file system structure} & \texttt{aspc=IDENT} & Name of an address space type (not used by \texttt{FIELD)}\tabularnewline
 &  & \texttt{type=TYPE} & Type of the referenced structure\tabularnewline
\texttt{FIELD} & \multirow{3}{0.2\textwidth}{Field is an offset to a structure within the same container} & \texttt{when=BOOL} & Pointer is valid when condition is true\tabularnewline
 &  & \texttt{size=INT} & Size of the referenced metadata\tabularnewline
 &  & \texttt{name=IDENT,expr=INT} & Name of an implicit pointer, its expression\tabularnewline
\hline 
\texttt{ADDRSPACE} & An address space & \texttt{name=IDENT} & Name of the address space type\tabularnewline
\hline 
\texttt{CHECK} & Constraint check & \texttt{expr=BOOL} & Condition for the structure's correctness\tabularnewline
\hline 
\texttt{VECTOR} & 1. Defines a vector type & \texttt{name=IDENT} & Name of the vector or the field name of the array\tabularnewline
 & \multirow{2}{0.2\textwidth}{2. Defines a flexible array field} & \texttt{type=TYPE} & Structure type of the contained elements\tabularnewline
 &  & \texttt{count=INT} & Number of elements in the vector\tabularnewline
 &  & \texttt{size=INT} & Size of the vector, in bytes\tabularnewline
 &  & \texttt{sentinel=BOOL} & Sentinel value that specifies the end of vector\tabularnewline
\hline 
\end{tabular}
\par\end{centering}
\begin{centering}
\medskip{}
\begin{minipage}[t]{0.9\textwidth}%
\texttt{IDENT} is a valid C identifier. \texttt{TYPE} is the type
name of a structure or vector type. \texttt{BOOL} and \texttt{INT}
are syntactically valid, dynamically scoped, C expressions that evaluate
to a boolean and integer type.%
\end{minipage}
\par\end{centering}
\centering{}\caption{\label{tab:annotation_language}Spiffy file system annotations.}
\end{table*}

\section{Annotation Language\label{sec:Annotation_Language}}

Spiffy uses annotations on C structures to specify the format of file-system
structures. We chose this approach to reduce duplication of structure
definitions. The annotations are defined using C preprocessor macros.
They are designed to be compatible with existing code by expanding
to empty code during normal compilation. Although many annotations
can be added to existing structures, sometimes we need to add new
structures or modify existing structures when they are a poor fit
for our needs.

%In \ref{subsec:Annotation-Effort}, we describe the effort needed to write annotations for modern file systems.


Table~\ref{tab:annotation_language} shows the list of annotations
supported by Spiffy. Each annotation is written using one or more
keywords, followed by their arguments. We now describe each annotation.

\paragraph{FSSTRUCT, FSSUPER}

These annotations are written by replacing the \texttt{struct} keyword
in a C structure with \texttt{FSSTRUCT} or \texttt{FSSUPER}. They
help distinguish file system metadata from in-memory file system structures
so that our compiler only parses C data structures marked with these
two annotations. The \texttt{FSSUPER} annotation identifies the root
of the file system. The \texttt{location} argument describes its physical
location as an offset (in bytes) from the beginning of the file system
image. For \texttt{FSSTRUCT}, the \texttt{location} argument optionally
specifies its placement constraint. 
The \texttt{name} argument is used by a descendant to reference
this structure (see \ref{subsec:Annotation-Design}). The \texttt{unit}
argument specifies the access unit of the structure, with \texttt{object} 
being the default unit.

The \texttt{base-when} argument enables supporting context-sensitive types. It defines a structure that is derived from a base structure when the condition is true. Conceptually, the derived structure is appended to the base structure, similar to the way inheritance is implemented in object oriented languages. Figure~\ref{fig:f2fs_inode} shows an example in which the F2FS inode structure is inherited by either a directory inode structure or a file inode structure, depending on the mode of the inode. The use of two derived inode structures allows using different types in the two structures. For example, we use a \texttt{dir\_block} pointer in the directory inode and a \texttt{data\_block} pointer in the file inode.

% (jsun): the example is no longer valid for F2fs
%Spiffy also supports polymorphism. For example, \texttt{ext3\_inode\_block} is defined as a vector of \texttt{ext3\_inode} structures, even though the actual type of each inode may be a directory or file inode.
Notice that the two arguments in the \texttt{FSSTRUCT} definition of \texttt{f2fs\_inode} reference the super block using the name \texttt{sb}. 
In addition, the \texttt{location} argument specifies its
placement constraint so that incorrect allocation will not result in clobbering
parts of the F2FS static metadata area. The \texttt{\$self} notation refers
to the address of the container (see Figure~\ref{fig:address-struct}). 

\begin{figure}[t]
\begin{lstlisting}
#define BLOCK_SIZE 1 << sb.log_blocksize
FSSTRUCT(location=$self.id >= sb.main_blkaddr,
         size=BLOCK_SIZE) f2fs_inode {
  __le16 i_mode;
  ...
};
typedef FSSTRUCT(base=struct f2fs_inode, 
         when=self.i_mode & S_IFDIR) {
  POINTER(aspc=block, type=dir_block)
  __le32 i_addr[DEF_ADDRS_PER_INODE];        
  POINTER(aspc=nid, type=dir_direct_block)
  __le32 i_dnid[2];
  ...
} f2fs_dir_inode;
typedef FSSTRUCT(base=struct f2fs_inode, 
	     when=self.i_mode & S_IFREG) {
  POINTER(aspc=block, type=data_block)
  __le32 i_addr[DEF_ADDRS_PER_INODE];            
  POINTER(aspc=nid, type=data_direct_block)
  __le32 i_dnid[2];
  ...
} f2fs_reg_inode;
\end{lstlisting}
\caption{\label{fig:f2fs_inode}Annotation for file and directory inode structures in F2FS}
\end{figure}
% (jsun): unfortunately in f2fs there is no array of inode since 
% f2fs inode itself is the size of a block
%VECTOR(name=ext3_inode_block, type=struct ext3_inode, size=BLOCK_SIZE);

\paragraph{POINTER, FIELD, ADDRSPACE}

The \texttt{POINTER} annotation is used to specify the address type
and the pointed-to type of a pointer. It allows fetching a structure
from disk and parsing it with the correct type information. As an
example, we annotate the \texttt{s\_journal\_inum }field in the Ext4
super block, shown in Figure~\ref{fig:ext4_super_block}, to indicate
that it points to an \texttt{ext4\_journal} type, in the file address
space.

\begin{figure}
\begin{lstlisting}
FSSUPER(name=sb, location=1024) 
				 ext4_super_block {
  __le32 s_blocks_count;     // # of blocks
  __le32 s_log_block_size;   // block size
  __le32 s_blocks_per_group; // blocks per group 
  __le16 s_inode_size;       // size of inode
  ...
  /* pointer to journal in file address space */
  POINTER(aspc=file, type=ext4_journal)
  __le32 s_journal_inum;     
  ...
  /* implicit pointer to group descriptors */         
  POINTER(name=s_block_group_desc, aspc=block, 
          type=ext4_group_desc_table,                  
          expr=self.s_log_block_size ? 1 : 2);
};
\end{lstlisting}
\caption{\label{fig:ext4_super_block}Annotated Ext4 super block.}
\end{figure}

File systems may use the same pointer field to reference different
types of metadata. The \texttt{when} argument is used to specify context-sensitive
pointers. For example, Figure~\ref{fig:btrfs_super_block} shows
that the Btrfs ``tree of tree'' root points to a B-tree leaf when
the level of the tree is 0, or else it points to a B-tree node. In
this case, two pointer annotations are needed to specify each of the
pointed-to types and their \texttt{when} expression. The \texttt{size} argument
is useful when the structure that contains the pointer also stores
the information about the size of the pointed-to structure. This may
be the case when the pointed-to structure is variable-sized or a data
block.

\begin{figure}
\begin{lstlisting}
ADDRSPACE(name=raid);
FSSUPER(name=sb, location=0x10000) 
                 btrfs_super_block {
  ...
  POINTER(aspc=raid, type=struct btrfs_node, 
          when=self.root_level > 0)     
  POINTER(aspc=raid, type=struct btrfs_leaf, 
          when=self.root_level == 0) 
  __le64  root; 
  ...
      u8  root_level;  /* depth of root tree */
  ...
} __attribute__ ((__packed__));
\end{lstlisting}
\caption{\label{fig:btrfs_super_block}Annotated Btrfs super block.}
\end{figure}

Spiffy supports implicit pointers with the \texttt{name-expr} argument,
which names a pointer and specifies an expression for computing the
address value. For example, Figure~\ref{fig:ext4_super_block} shows
that we added an implicit field to the end of the Ext4 super block,
because it does not have a pointer field to the block group descriptor
table. The descriptor table is located at block 2 if the block size
is 1024 bytes, or block 1 for every other block size.

The \texttt{FIELD} annotation is similar to a pointer, but it is used
to specify offset fields that reference a structure within the \emph{same
}container. Unlike a file system pointer, a field access does not
require fetching data from disk, and hence it does not require an
address space.

The \texttt{ADDRSPACE} annotation specifies an address space for a
pointer type. Figure~\ref{fig:btrfs_super_block} shows that the
Btrfs pointers have a raid address type. In Section~\ref{sec:Implementation},
we describe how the annotation developer implements this annotation.

\paragraph{VECTOR}

The \texttt{VECTOR} annotation helps specify variable-length arrays
of structures. It can be placed inside or outside structure definitions.
When placed inside, it defines an implicit field of a structure. When
placed outside, it defines a new type, such as the \texttt{ext4\_dir\_block}
structure in Figure~\ref{fig:ext4_extent_block}. The size of the vector
can be specified using any of the \texttt{count}, \texttt{size} or
\texttt{sentinel} arguments. The \texttt{size} argument is useful
when the elements are variable-sized and that the number of elements
cannot be easily deduced. The \texttt{sentinel} argument specifies
a boolean condition for determining the last element of a vector.
All combinations of the three arguments are valid, and parsing ends
as soon as one of the stopping conditions are met. Vector types have
access units but the compiler can automatically deduce this information
based on the access units of their elements. A vector that contains
objects is a container (e.g., an inode block), and a vector that contains
containers or extents is an extent (e.g., inode table). 

\paragraph{CHECK}

The \texttt{CHECK} annotation allows specification of arbitrary constraints
associated with a structure. These checks are performed both after parsing a structure, and before serializing it. The annotation acts as an assertion, which upon failure, results in a parsing or a serialization error. Figure~\ref{fig:ext4_extent_block} shows an example where the \texttt{CHECK} annotation is used to verify that the extent header contains the correct magic number.

\subsection{Ext4 \label{subsec:Ext4}}

The Linux Ext4 file system is the most popular Linux file system. Unlike its
predecessor Ext3, it uses extent-based allocation instead of block-based allocation,
which significantly reduces metadata block usage for contiguous allocations.
We have modified and added some Ext4 data structures so that they
can can be specified correctly. For backward compatibility, the Ext4
developers decided to leave the \texttt{i\_block} field of the inode structure
definition alone, although the space it occupies is now used for 
an extent tree. We redefined an Ext4 inode so that it now properly contains 
a extent header followed by four extent entries. We also support Ext3's block-based allocation scheme, which is not shown here for brevity. We also added a definition for the extent leaf blocks, shown in Figure~\ref{fig:ext4_extent_block}, which was omitted in the original header file.
\begin{figure}
\begin{lstlisting}
FSSTRUCT(name=eh) ext4_extent_header {
	__u16	eh_magic, eh_entries;	
	__u16	eh_max,   eh_depth;	
	__u32	eh_generation;
	CHECK(expr=self.eh_magic == EXT4_EXT_MAGIC);
};
FSSTRUCT(size=BLOCK_SIZE) ext4_extent_leaf {
    struct ext4_extent_header eb_hdr;
    FIELD(count=self.eb_hdr.eh_entries)
    struct ext4_extent eb_extent[];
};
VECTOR(name=ext4_dir_block, size=BLOCK_SIZE,
	   type=struct ext4_dir_entry);
\end{lstlisting}
\caption{\label{fig:ext4_extent_block}Annotations for Ext4 extent header and leaf, and Ext4 directory block}
\end{figure}

\subsection{Btrfs\label{subsec:Btrfs}}

Btrfs is a copy-on-write file system that stores in data structures
in a number of B-trees~\cite{btrfs2013}. Each B-tree uses two types
of containers, an internal node that contains a sorted list of key-pointer
pairs, and a leaf node that contains a set of keys and their associated
metadata objects. The internal node structure is relatively simple,
so we describe the annotation for a leaf node. Btrfs places all of
the file system's metadata objects (e.g., inode, directory entries)
in reverse order, starting from the end of the B-tree leaf block,
as shown in Figure~\ref{fig:Btrfs-B-tree-leaf}. For each metadata
object, there is a corresponding \texttt{btrfs\_item} object that
stores the offset and size of the metadata object. For example, \texttt{items{[}0{]}}
stores the offset and size for \texttt{data{[}0{]}.}

Figure~\ref{fig:btrfs_item} shows the annotated Btrfs leaf node
(\texttt{btrfs\_leaf}), containing a vector of \texttt{btrfs\_item}
structures. The \texttt{btrfs\_item} structure defines implicit fields
with the \texttt{FIELD} annotation. These fields use the \texttt{when}
expression to point to all the different types of metadata objects that
can be stored in a leaf object. The \texttt{offset} field is an offset
to a metadata object from the end of the header field of \texttt{btrfs\_leaf},
and so we must add this value to obtain the offset from the beginning
of the container (the leaf node).

\begin{figure}
\centering{}\includegraphics[width=1\columnwidth]{figures/btrfs_leaf}\caption{\label{fig:Btrfs-B-tree-leaf}Btrfs leaf node layout.}
\end{figure}

\begin{figure}
\begin{lstlisting}
FSSTRUCT(unit=container,
         size=sb.leafsize) btrfs_leaf {
  struct btrfs_header header;
  VECTOR(name=items, type=struct btrfs_item,
         count=self.header.nritems)
};

#define METADATA_LOCATION \
  (sizeof(struct btrfs_header)+self.offset)
FSSTRUCT() btrfs_item { 	
  struct btrfs_disk_key key; 	
  __le32 offset, size;
  FIELD(name=metadata, size=self.size,
        type=struct btrfs_file_extent_item,  
        expr=METADATA_LOCATION,
        when=self.key.type == 
             BTRFS_EXTENT_DATA_KEY)
  FIELD(name=metadata, size=self.size,
        type=struct btrfs_inode_item, 
        expr=METADATA_LOCATION,
        when=self.key.type == 
             BTRFS_INODE_ITEM_KEY);
  /* followed by 15 more implicit fields,
     each with a different type */
};
\end{lstlisting}

\caption{\label{fig:btrfs_item}Btrfs leaf node and item structure.}
\end{figure}

\subsection{F2FS\label{subsec:F2FS}}

\begin{figure}
\begin{lstlisting}
typedef FSSTRUCT(name=cphdr, rank=container,
  base=struct f2fs_checkpoint) {
  POINTER(repr=block, type=f2fs_orphan_blocks,
  expr=$self.id + 1, 
  when=self.ckpt_flags & CP_ORPHAN_PRESENT_FLAG);
  ...
} f2fs_checkpoint_header;
\end{lstlisting}

\caption{\label{fig:checkpoint-header}F2FS checkpoint header annotations.}
\end{figure}

F2FS is a relatively new, log-structured file system optimized for NAND flash storage devices. Its on-disk layout is partitioned into fixed-sized segments composed of a set of contiguous blocks, with each segment sized in units of the SSD's erase block size to minimize wear. The file system contains five static metadata areas, and one main area for data blocks and dynamically allocated metadata, such as the inode shown in Figure~\ref{fig:f2fs_inode}.
The static metadata area consists of a pair of checkpoint packs, as shown in Figure \ref{fig:F2FS-checkpoint-packs}, and various lookup tables for space and pointer management.

A unique challenge with annotating F2FS is its use of heterogeneous extents, i.e., extents with different types of metadata blocks. F2FS has a super block to the first checkpoint pack, shown in Figure~\ref{fig:F2FS-checkpoint-packs}. The remaining metadata blocks must be referenced by using implicit pointers. However, the block addresses of these metadata
blocks depend on the address of the checkpoint header. Therefore, we use a special variable, \texttt{\$self}, to allow implicit pointers to specify metadata blocks that exist at certain block offsets from the current container. Figure~\ref{fig:checkpoint-header} shows the annotation for the implicit pointer that points to a vector of orphan blocks.


\end{document}

